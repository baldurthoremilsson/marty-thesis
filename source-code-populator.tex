\begin{lstlisting}[caption={populator.py}]
# -*- coding: utf-8 -*-

import logging

from dbobjects import Table, Column


class HistoryPopulator(object):

    def __init__(self, con, logger=None):
        self.con = con
        self.update_id = None
        if logger:
            self.logger = logger
        else:
            self.logger = logging.getLogger()
            self.logger.addHandler(logging.NullHandler())

    def create_tables(self):
        self.logger.info('creating tables')

        with self.con.cursor() as curs:
            # marty_updates
            curs.execute("""
            CREATE TABLE IF NOT EXISTS marty_updates(
                id SERIAL PRIMARY KEY,
                time TIMESTAMP DEFAULT current_timestamp NOT NULL,
                mastertime TIMESTAMP NOT NULL
            )
            """)

            # marty_schemas
            curs.execute("""
            CREATE TABLE IF NOT EXISTS marty_schemas(
                _ctid tid NOT NULL,
                oid oid NOT NULL,
                name name NOT NULL,
                start integer REFERENCES marty_updates(id) NOT NULL,
                stop integer REFERENCES marty_updates(id)
            )
            """)

            # marty_tables
            curs.execute("""
            CREATE TABLE IF NOT EXISTS marty_tables(
                _ctid tid NOT NULL,
                oid oid NOT NULL,
                name name NOT NULL,
                schema oid NOT NULL,
                internal_name name NOT NULL,
                start integer REFERENCES marty_updates(id) NOT NULL,
                stop integer REFERENCES marty_updates(id)
            )
            """)

            # marty_columns
            curs.execute("""
            CREATE TABLE IF NOT EXISTS marty_columns(
                _ctid tid NOT NULL,
                table_oid oid NOT NULL,
                name name NOT NULL,
                number int2 NOT NULL,
                type name NOT NULL,
                length int4 NOT NULL,
                internal_name name NOT NULL,
                start integer REFERENCES marty_updates(id) NOT NULL,
                stop integer REFERENCES marty_updates(id)
            )
            """)

    def update(self, mastertime):
        with self.con.cursor() as curs:
            curs.execute("""
            INSERT INTO marty_updates(mastertime) VALUES(%s) RETURNING id
            """, (mastertime,))
            self.update_id = curs.fetchone()[0]
            self.logger.debug('new update id {}'.format(self.update_id))

    def add_schema(self, schema):
        self.logger.info('adding schema {}'.format(schema.name))

        with self.con.cursor() as curs:
            curs.execute("""
            INSERT INTO marty_schemas(_ctid, oid, name, start) VALUES(%s, %s, %s, %s)
            """, (schema.ctid, schema.oid, schema.name, self.update_id))

    def remove_schema(self, ctid):
        self.logger.info('removing schema {}'.format(ctid))

        with self.con.cursor() as curs:
            curs.execute("""
            UPDATE marty_schemas SET stop = %s WHERE _ctid = %s
            """, (self.update_id, ctid))

    def add_table(self, table):
        self.logger.info('adding table {}'.format(table.long_name))

        update = self.update_id
        table.update = update
        with self.con.cursor() as curs:
            curs.execute("""
            INSERT INTO marty_tables(_ctid, oid, name, schema, internal_name, start)
            VALUES(%s, %s, %s, %s, %s, %s)
            """, (table.ctid, table.oid, table.name, table.schema.oid, table.internal_name, self.update_id))

            self.logger.debug(curs.query)

            for column in table.columns:
                self.add_column(column)

    def remove_table(self, ctid):
        self.logger.info('removing table {}'.format(ctid))

        with self.con.cursor() as curs:
            curs.execute("""
            UPDATE marty_tables SET stop = %s WHERE _ctid = %s
            """, (self.update_id, ctid))

    def add_column(self, column):
        self.logger.info('adding column {} to {}'.format(column.name, column.table.long_name))

        with self.con.cursor() as curs:
            curs.execute("""
            INSERT INTO marty_columns(_ctid, table_oid, name, number, type, length, internal_name, start)
            VALUES(%s, %s, %s, %s, %s, %s, %s, %s)
            """, (column.ctid, column.table.oid, column.name, column.number, column.type,
                column.length, column.internal_name, self.update_id))

            self.logger.debug(curs.query)

    def remove_column(self, ctid):
        self.logger.info('removing column {}'.format(ctid))

        with self.con.cursor() as curs:
            curs.execute("""
            UPDATE marty_columns SET stop = %s WHERE _ctid = %s
            """, (self.update_id, ctid))

    def create_table(self, table):
        self.logger.info('creating table {}'.format(table.internal_name))

        with self.con.cursor() as curs:
            cols = ','.join('\n  {} {}'.format(column.internal_name, column.type) for column in table.internal_columns)
            curs.execute('CREATE TABLE {}({})'.format(table.internal_name, cols))

            self.logger.debug(curs.query)

            curs.execute('SELECT oid FROM pg_class WHERE relname = %s', (table.internal_name,))
            table_oid, = curs.fetchone()

            for column in table.columns:
                curs.execute("""
                UPDATE pg_attribute
                SET atttypmod = %s
                WHERE attrelid = %s AND attname = %s
                """, (column.length, table_oid, column.internal_name))

    def add_data_column(self, column):
        with self.con.cursor() as curs:
            curs.execute("""
            ALTER TABLE {} ADD COLUMN {} {}
            """.format(column.table.internal_name, column.internal_name, column.type))

            curs.execute('SELECT oid FROM pg_class WHERE relname = %s', (column.table.internal_name,))
            table_oid, = curs.fetchone()

            curs.execute("""
            UPDATE pg_attribute
            SET atttypmod = %s
            WHERE attrelid = %s AND attname = %s
            """, (column.length, table_oid, column.internal_name))

    def fill_table(self, table):
        self.logger.info('filling table {}'.format(table.internal_name))

        table_name = table.internal_name
        column_names = ', '.join(column.internal_name for column in table.internal_columns)
        value_list = ', '.join('%s' for column in table.internal_columns)
        query = 'INSERT INTO {}({}) VALUES({})'.format(table_name, column_names, value_list)

        with self.con.cursor() as curs:
            for line in table.data():
                values = list(line)
                values.extend([self.update_id, None])
                curs.execute(query, values)

                self.logger.debug(curs.query)

    def insert(self, table, block, offset, row):
        self.logger.info('inserting to table {}'.format(table.internal_name))
        table_name = table.internal_name
        column_names = ', '.join(column.internal_name for column in table.internal_columns)
        value_list = ', '.join('%s' for column in table.internal_columns)
        query = 'INSERT INTO {}({}) VALUES({})'.format(table_name, column_names, value_list)

        values = ['({},{})'.format(block, offset)] + list(row) + [self.update_id, None]
        with self.con.cursor() as curs:
            curs.execute(query, values)
            self.logger.debug(curs.query)

    def delete(self, table, block, offset):
        self.logger.info('deleting from table {}'.format(table.internal_name))
        query = 'UPDATE {} SET stop = %s WHERE data_ctid = %s'.format(table.internal_name)
        values = [self.update_id, '({},{})'.format(block, offset)]
        with self.con.cursor() as curs:
            curs.execute(query, values)
            self.logger.debug(curs.query)

    def delete_all(self, table):
        query = 'UPDATE {} SET stop = %s WHERE stop IS NULL'.format(table.internal_name)
        values = (self.update_id,)
        with self.con.cursor() as curs:
            curs.execute(query, values)

    def get_table(self, ctid):
        with self.con.cursor() as curs:
            curs.execute("""
            SELECT _ctid, oid, name, internal_name
            FROM marty_tables
            WHERE _ctid = %s""", (ctid,))
            row = curs.fetchone()
            if not row:
                return
            ctid, oid, name, internal_name = row
            return Table(None, ctid, oid, name, internal_name=internal_name)

    def get_column(self, ctid):
        with self.con.cursor() as curs:
            curs.execute("""
            SELECT _ctid, table_oid, name, number, type, length, internal_name
            FROM marty_columns
            WHERE _ctid = %s""", (ctid,))
            row = curs.fetchone()
            if not row:
                return
            ctid, table_oid, name, number, type, length, internal_name = row
            return Column(None, ctid, name, number, type, length, internal_name=internal_name)


class ClonePopulator(object):

    def __init__(self, con, update, history_coninfo, logger=None):
        self.con = con
        self.update = update
        self.history_coninfo = history_coninfo
        if logger:
            self.logger = logger
        else:
            self.logger = logging.getLogger()
            self.logger.addHandler(logging.NullHandler())

    def initialize(self):
        with self.con.cursor() as curs:
            curs.execute('CREATE SCHEMA IF NOT EXISTS marty')
            curs.execute('CREATE EXTENSION IF NOT EXISTS dblink')
            curs.execute("""
            CREATE TABLE marty.bookkeeping(
                view_name name UNIQUE,
                local_table name,
                cached boolean DEFAULT false,
                coldef text,
                remote_select_stmt text,
                temp_table_def text
            )
            """)

            curs.execute("""
            CREATE FUNCTION coninfo() RETURNS text AS $$
            BEGIN
                RETURN '{coninfo}';
            END;
            $$ LANGUAGE plpgsql;
            """.format(coninfo=self._dblink_connstr()))

            curs.execute("""
            CREATE FUNCTION view_select(my_view_name text) RETURNS SETOF RECORD AS $$
            DECLARE
                view_info RECORD;
            BEGIN
                SELECT * FROM marty.bookkeeping WHERE view_name = my_view_name INTO view_info;
                IF NOT view_info.cached THEN
                    RAISE NOTICE 'fetching %', view_info.view_name;
                    EXECUTE ' INSERT INTO ' || view_info.local_table ||
                            ' SELECT ' || view_info.coldef ||
                            ' FROM dblink(''' || coninfo() || ''', ''' || view_info.remote_select_stmt || ''')'
                            ' AS ' || view_info.temp_table_def;
                    UPDATE marty.bookkeeping SET cached = true WHERE view_name = my_view_name;
                END IF;
                RETURN QUERY EXECUTE 'SELECT ' || view_info.coldef || ' FROM ' || view_info.local_table;
            END;
            $$ LANGUAGE plpgsql;
            """)

    def _dblink_connstr(self):
        parts = {
            'host': 'host={}',
            'port': 'port={}',
            'user': 'user={}',
            'password': 'password={}',
            'database': 'dbname={}',
        }
        return ' '.join(parts[key].format(value) for key, value in self.history_coninfo.iteritems())

    def create_schema(self, schema):
        self.logger.info('Creating schema {}'.format(schema.name))
        with self.con.cursor() as curs:
            curs.execute('CREATE SCHEMA IF NOT EXISTS {}'.format(schema.name))

    def create_table(self, table):
        self.logger.info('Creating table {}'.format(table.long_name))

        # Create table for local data
        table.update = self.update
        query = 'CREATE TABLE marty.{table}({cols})'
        cols = ','.join('\n  "{name}" {type}'.format(name=column.name, type=column.type) for column in table.columns)
        with self.con.cursor() as curs:
            curs.execute(query.format(table=table.internal_name, cols=cols))
            for column in table.columns:
                curs.execute("""
                UPDATE pg_attribute
                SET atttypmod = %(column_length)s
                WHERE attrelid = %(table_name)s::regclass::oid AND attname = %(column_name)s
                """, {'column_length': column.length, 'table_name': 'marty.{}'.format(table.internal_name), 'column_name': column.name})


            # Create view that combines local and remote data
            my_cols = ', '.join(['"{}"'.format(col.name) for col in table.columns])
            temp_columns = ['"{name}" {type}'.format(name=col.name, type=col.type) for col in table.columns]
            temp_table_def = 't1({columns})'.format(columns=', '.join(temp_columns))

            view_query = """
            CREATE VIEW {view_name}
            AS SELECT {cols} FROM view_select('{view_name}')
            AS {tabledef};
            """
            curs.execute(view_query.format(view_name=table.long_name, cols=my_cols, tabledef=temp_table_def))

            bookkeeping_query = """
            INSERT INTO marty.bookkeeping(view_name, local_table, coldef, remote_select_stmt, temp_table_def)
            VALUES(%(view_name)s, %(local_table)s, %(coldef)s, %(remote_select_stmt)s, %(temp_table_def)s);
            """
            local_cols = ', '.join(['"{}"'.format(col.name) for col in table.columns])
            internal_cols = ', '.join([col.internal_name for col in table.columns])
            remote_select_stmt = 'SELECT {cols} FROM {table} WHERE start <= {update} and (stop IS NULL or stop > {update})'
            bookkeeping_values = {
                'view_name': table.long_name,
                'local_table': 'marty.' + table.internal_name,
                'coldef': local_cols,
                'remote_select_stmt': remote_select_stmt.format(cols=internal_cols, table=table.internal_name, update=self.update),
                'temp_table_def': temp_table_def,
            }
            curs.execute(bookkeeping_query, bookkeeping_values)

            trigger_queries_values = {
                'trigger_name': table.long_name.replace('.', '_'),
                'local_table': 'marty.' + table.internal_name,
                'local_columns': my_cols,
                'new_values_insert': ', '.join(['NEW.' + col.name for col in table.columns]),
                'new_values_update': ', '.join(['"{name}" = NEW.{name}'.format(name=col.name) for col in table.columns]),
                'old_values': ' AND '.join(['"{name}" = OLD.{name}'.format(name=col.name) for col in table.columns]),
                'view_name': table.long_name,
            }

            # Create insert trigger for view
            insert_query = """
            CREATE FUNCTION {trigger_name}_insert() RETURNS trigger AS $$
                BEGIN
                    INSERT INTO {local_table}({local_columns}) VALUES({new_values_insert});
                    RETURN NEW;
                END;
            $$ LANGUAGE plpgsql;

            CREATE TRIGGER {trigger_name}_insert_trigger
            INSTEAD OF INSERT ON {view_name}
            FOR EACH ROW EXECUTE PROCEDURE {trigger_name}_insert();
            """
            curs.execute(insert_query.format(**trigger_queries_values))

            # Create update trigger for view
            update_query = """
            CREATE FUNCTION {trigger_name}_update() RETURNS trigger AS $$
                BEGIN
                    UPDATE {local_table} SET {new_values_update} WHERE {old_values};
                    RETURN NEW;
                END;
            $$ LANGUAGE plpgsql;

            CREATE TRIGGER {trigger_name}_update_trigger
            INSTEAD OF UPDATE ON {view_name}
            FOR EACH ROW EXECUTE PROCEDURE {trigger_name}_update();
            """
            curs.execute(update_query.format(**trigger_queries_values))

            # Create delete trigger for view
            delete_query = """
            CREATE FUNCTION {trigger_name}_delete() RETURNS trigger AS $$
                BEGIN
                    DELETE FROM {local_table} WHERE {old_values};
                    RETURN OLD;
                END;
            $$ LANGUAGE plpgsql;

            CREATE TRIGGER {trigger_name}_delete_trigger
            INSTEAD OF DELETE ON {view_name}
            FOR EACH ROW EXECUTE PROCEDURE {trigger_name}_delete();
            """
            curs.execute(delete_query.format(**trigger_queries_values))
\end{lstlisting}