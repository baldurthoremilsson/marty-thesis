\section{The history database}

\begin{wrapfigure}{r}{0.5\textwidth}
  \vspace{-20pt}
  \begin{center}
    \includegraphics[width=0.48\textwidth]{img/architecture-history}
  \end{center}
  \vspace{-20pt}
  \caption{The history part of the architecture}
  \vspace{-10pt}
\end{wrapfigure}

The history database contains information about the schema of the master database and a copy of its data.
It contains multiple versions of the schema and data from the master, each transaction that alters the master database creates a history version.
It is used as a reference when the user creates a new clone database; the schema of the clone is created according to the information in the history database and its data is copied from the history.

The history database is created when the slave instance has been configured correctly.
It is a standard Postgres database that must have the same minor version as the master database.
That means that if the master runs on Postgres version 9.3.3 the history version must start with 9.3, see the Postgres documentation for further reference. % TODO add reference?

\subsection{Schema information}
To initialize the history database the administrator runs the script \textit{history.py}.
It starts by creating the \textit{schema information tables} which Marty uses to store information about the schema of the master database:

\begin{description}
  \item[marty\_schemas]
    Contains the name and ID of all schemas in the slave database.
  \item[marty\_tables]
    Contains the name and ID of the tables in the slave database and a reference to the schema they are in.
    It also stores the \textit{internal name} which is used for the data table in the history and clone databases.
  \item[marty\_columns]
    Contains the name, number, type and length of each column in the tables.
    It also stores an internal name for the column that is used in the data tables in the history and clone databases.
\end{description}

Tables \ref{tbl:marty-schemas} to \ref{tbl:marty-columns} show information about the columns of these tables.
Note that each table contains a column called \textit{\_ctid}, see Chapter \ref{ch:implementation-history-ctid} for more information about the value in this column.
The tables also have two columns called \textit{start} and \textit{stop}.
These columns define which \textit{history version} each row is part of, see chapter \ref{ch:implementation-history-versions} for further information.
Figure \ref{fig:schema-information-example} shows an example of the contents of the schema information tables.

\begin{table}[h]
  \centering
  \textbf{marty\_schemas}
  \begin{tabularx}{\textwidth}{llX}
    \textit{Column} & \textit{Type} & \textit{Description} \\
    \midrule
    \_ctid & tid & A reference to the pg\_namespace table in the slave database \\
    oid & oid & The ID of the schema in the slave database \\
    name & name & The name of the schema \\
    start & integer & First version where this schema is present in the database \\
    stop & integer & First version where this schema stops being present in the database \\
  \end{tabularx}
  \caption{The columns of the marty\_schemas table}
  \label{tbl:marty-schemas}
\end{table}

\begin{table}[h]
  \centering
  \textbf{marty\_tables}
  \begin{tabularx}{\textwidth}{llX}
    \textit{Column} & \textit{Type} & \textit{Description} \\
    \midrule
    \_ctid & tid & A reference to the pg\_class table in the slave database \\
    oid & oid & The ID of the table in the slave database \\
    name & name & The name of the table \\
    schema\_oid & oid & A reference to the schema that this table belongs to \\
    internal\_name & name & The name of the data table, see Chapter \ref{ch:implementation-history-data} \\
    start & integer & First version where this table is present in the database \\
    stop & integer & First version where this table stops being present in the database \\
  \end{tabularx}
  \caption{The columns of the marty\_tables table}
  \label{tbl:marty-tables}
\end{table}

\begin{table}[h]
  \centering
  \textbf{marty\_columns}
  \begin{tabularx}{\textwidth}{llX}
    \textit{Column} & \textit{Type} & \textit{Description} \\
    \midrule
    \_ctid & tid & A reference to the pg\_attribute table in the slave database \\
    table\_oid & oid & A reference to the table this column is in \\
    name & name & The name of the column \\
    number & int2 & The index of this column in the table (is it the first, second, third etc.) \\
    type & name & The type of the column (int, text, boolean etc.) \\
    length & int4 & The length of the column. Some data types require the user to specify a length, such as character columns, e.g. char(26). Columns without length usually have the value -1 \\
    internal\_name & name & The name of this column in the data table, see Chapter \ref{ch:implementation-history-data} \\
    start & integer & First version where this column is part of the table \\
    stop & integer & First version where this column stops being part of the table \\
  \end{tabularx}
  \caption{The columns of the marty\_tables table}
  \label{tbl:marty-columns}
\end{table}

Postgres uses \textit{system catalogs} to store information about the schema of its databases.
They are used internally, e.g. when Postgres reads from or writes to tables.
Information such as which file in the database cluster represents which table and the column names and types of each table are stored in the system catalogs.
They are ordinary tables which are stored in a schema called \textit{pg\_catalogs} and can by queried by a user just as any other table.

Marty reads information from four system catalogs in the slave database:

\begin{description}
  \item[pg\_namespace]
    Contains information about the schemas in the slave database.
    Information from this catalog is saved in \textit{marty\_schemas} in the history database.
  \item[pg\_class]
    Contains information about the tables in the slave database.
    Information from this catalog is saved in \textit{marty\_tables} in the history database.
  \item[pg\_attribute]
    Contains information about the columns of the tables in the slave database.
    Information from this catalog is saved in \textit{marty\_columns} in the history database.
  \item[pg\_type]
    Contains additional information about the columns.
    Marty reads the name of the column type from this table (integer, text etc.) and stores it in \textit{marty\_columns} along with the other column related information.
\end{description}

The history.py script creates the schema information tables and then it populates them with information about the schema of the slave database.
This is of course also the schema of the master database so the history database really contains information about the master.

Marty inspects the schema of the slave database before any WAL records have been applied to it.
It writes the schema information to the history database, which creates the first history version, see Chapter \ref{ch:implementation-history-versions} for more about history versions.
Marty then starts the WAL replay in the slave database and reads the replay log.

When a new schema is created the WAL contains an \textit{insert} heap record for the pg\_namespace table.
Marty sees this in the replay log and queries the slave about this new schema.
It then creates a new history version that includes it.
The same happens when a new table is created or a new column is added to a table; a new history version is created that includes the new table or column.

If a schema, table, or a column is altered, e.g. when they are renamed, the WAL has an \textit{update} heap record.
Similarly when a schema, table or column is dropped the WAL has a \textit{delete} heap record.
Marty sees this in the replay log and creates new history versions with the altered database schema.

\begin{figure}[h!]
  \centering
    \includegraphics[width=0.75\textwidth]{schema-information-example}
  \caption{Example of the contents of the schema information tables}
  \label{fig:schema-information-example}
\end{figure}

See Figure \ref{fig:schema-information-example} for an example of the contents of the schema information tables.

In the example the master database has a schema called \textit{myschema}.
Its object identifier (oid) is 2200.
The \textit{pg\_namespace} system catalog in the slave database stores information about this schema in a row with the ctid value of (0,5).
This is reflected in the \textit{marty\_schemas} table in the history database.

The master contains one table, \textit{persons}, which belongs to myschema.
Information about this table is stored in \textit{marty\_tables}.
Its oid is 16385 and the slave stores information about this table in a row with ctid value of (0,46) in the \textit{pg\_class} system catalog.
Its internal name is \textit{data\_myschema\_persons\_1}, see Chapter \ref{ch:implementation-history-data} for information about internal names of tables and columns.

The persons table has two columns, \textit{age} of type \textit{integer} or \textit{int4}, and \textit{name} of type \textit{text}.
The slave stores information about these columns in the \textit{pg\_attribute} system catalog in rows with ctid values of (39,32) and (39,33).
The age column is the first column in the table, the name column is the second one.
This is reflected in the \textit{number} value in \textit{marty\_columns}.
Columns of type integer and text do not use the length attribute, so Postgres puts -1 there.
The internal names of the columns are \textit{data\_age\_1} and \textit{data\_name\_1}.

The schema, table and columns are all part of the first history version.
This is reflected in the columns \textit{start} in the schema information tables.
They have not yet been altered or dropped and are thus part of every history version after the first one, which is reflected with the NULL value in the \textit{stop} column.
See Chapter \ref{ch:implementation-history-versions} for information about history versions.
\subsection{Data tables}
\label{ch:implementation-history-data}
The history database stores a copy of all the data from the slave database tables.
Marty creates a table in the history database for each table in the slave.
This happens immediately after Marty has saved the schema information about the original table to the history database.
These tables are the \textit{data tables} of the history database.

The schema of a data table is not identical to the schema of the original table.
Marty adds three columns for metadata; \textit{data\_ctid}, \textit{start} and \textit{stop}.
These columns are also present in the schema information tables (where \textit{data\_ctid} is called \textit{\_ctid}) where they serve the same purpose, see Chapters \ref{ch:implementation-history-ctid} and \ref{ch:implementation-history-versions} for more information.

When a column is added to the original table in the slave database Marty also adds it to the corresponding data table.
If a column is dropped from the original table Marty still keeps the column in the data table.
This is necessary because Marty keeps old versions of the data in the table and must therefore keep all columns that have been added to the original table, regardless of whether they have since been dropped or not.

When a row is deleted from the original table Marty marks it as deleted in the data table.
The row is still kept in the data table as part of an old history version because the data must still be accessible.
When a row is updated in the original table Marty inserts a new row into the data table with the updated values and marks the old row as deleted.
This behaviour is similar to the \textit{multiversion concurrency control} (MVCC) that Postgres uses to allow more than one transaction to use the same table at the same time.

Marty does not copy any constraints from the original tables to the data tables.
The only data that the data tables contain has been copied from the original tables in the slave database.
This data conforms to the constraints of the original tables and it is therefore not necessary to replicate the constraints in the data tables.
The constraints would also cause trouble, e.g. when Marty stores updated rows from a table with a unique constraint.
If the values of the unique columns are not updated but only values in other columns then the data table could not store both the old and the new versions of the rows, as the values in the unique columns would be the same in both rows.
For these reasons Marty does not copy any constraints from the original tables.

\begin{figure}[h!]
  \centering
    \includegraphics[width=0.6\textwidth]{history-data-table}
  \caption{An example of the schema of a data table in the history database}
  \label{fig:history-data-table-2}
\end{figure}

Marty creates all data tables in the master schema of the history database.
To avoid naming conflicts new names are creates for them: \textit{data\_[schema]\_[table]\_[version]}.
They all have the prefix \textit{data\_} followed by the name of the schema they are part of in the slave database.
Next comes the name of the original table and the name ends with the ID of the history version where this table was created.
This makes it possible for Marty to keep all the data tables in one schema in the history database, even if more than one table from different schemas share the same name in the slave database.
It also handles name reuse when a table is dropped and another table is created with the same name.
These two tables are not part of the same history version and thus the names of the data tables are different.

\begin{figure}[h]
  \centering
    \includegraphics[width=0.6\textwidth]{history-data-table-contents}
  \caption{An example of the data in a data table in the history database}
  \label{fig:history-data-table-contents}
\end{figure}

The columns of the data tables use similar names.
They have the \textit{data\_} prefix and are suffixed with the history version where they were added to the table.
See Figure \ref{fig:history-data-table-2} for an example of the schema of a data table in the history database.
Figure \ref{fig:history-data-table-contents} shows an example of the contents of a data table.
\subsection{CTID columns}
\label{ch:implementation-history-ctid}
The schema information tables have a column called \textit{\_ctid} and the data tables have an identical column called \textit{data\_ctid}.
The values in these columns reference the values in the \textit{ctid} columns in the corresponding tables in the slave database.
For the schema information table these are the system catalogs and for the data tables the corresponding tables are the original tables in the slave database.

The ctid column is a hidden system column that Postgres adds to all tables when they are created.
It is of type \textit{tid}, or tuple identifier, which is a pair of numbers that identify the physical location of the row inside the relation file in the database cluster.
The relation files are made up of \textit{blocks} which store the contents of the relations.
Each block contains one or more \textit{tuples} which store the actual data of the table rows.
The tuple identifier consists of a \textit{block index} and a \textit{tuple index}.
The block index is zero-based while the tuple index is one-based so the first tuple ID is (0,1).

\begin{lstlisting}[caption={WAL update and delete example},label={lst:wal-update-delete},numbers=left,xleftmargin=2em]
LOG:  REDO @ 0/60080CC; LSN 0/6008198: prev 0/600806C; xid 735; len 37; bkpb0: Heap - update: rel 1663/16384/16385; tid 0/1 xmax 735 ; new tid 0/2 xmax 0
LOG:  REDO @ 0/600822C; LSN 0/6008264: prev 0/6008204; xid 737; len 26: Heap - delete: rel 1663/16384/16385; tid 0/3 KEYS_UPDATED
\end{lstlisting}

The values in the ctid column can be used to identify each row in the database.
Marty saves this value in the schema information tables and data tables to be able to identify deleted and updated rows.
The WAL replay log from the slave instance contains the tuple identifier of the rows that are updated or deleted in the WAL records.
Marty uses the tid value to locate the correct row in the schema information tables or data tables in the history database and marks these rows as deleted.
See Listing \ref{lst:wal-update-delete} for an example.
In the example the update record in the first line inserts a new row with tid value of (0,2) to replace the old values in row with tid value of (0,1).
The delete record in the second line deletes a row with tid value of (0,3).

Note that the value of the ctid column can change, e.g. when Postgres \textit{vacuums} the table or when a row is updated.
Therefore it is not a viable long-term key to identify the row.
This does not affect the way Marty uses it because it only needs to know the tid value of the row until it changes.
When it is changed in the slave database Marty reflects this change in the history database and saves the new tid value.

Figure \ref{fig:schema-information-example} shows an example of the values in the \textit{\_ctid} column in the schema information tables.
The table \textit{marty\_schemas} contains information that Marty reads from the \textit{pg\_namespace} system catalog.
In the example this table contains one row with values from the row with ctid (0,5) in \textit{pg\_namespace}.
The table \textit{marty\_tables} contains one row with information from the \textit{pg\_class} system catalog.
It contains values from the row with ctid (0,46) in \textit{pg\_class}.
The table \textit{marty\_columns} contains two rows with information from the \textit{pg\_attribute} catalog.
They contain values from the rows with ctids (39,32) and (39,33) in \textit{pg\_attribute}.

Figure \ref{fig:history-data-table-contents} shows an example of a value in the \textit{data\_ctid} column in a data table.
The data table is called \textit{data\_myschema\_persons\_1} and the row it contains has values from a row in the original table in the master database that has the ctid value (0,1).
\subsection{History versions}
\label{chapter:implementation-history-versions}