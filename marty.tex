\documentclass[a4paper,12pt,twoside,BCOR=10mm,listof=totoc]{scrbook}

% Packages
\usepackage{ucs}
\usepackage[utf8x]{inputenc}
\usepackage[icelandic, english]{babel}
\usepackage{t1enc}
\usepackage{graphicx}
\usepackage[intoc]{nomencl}
\usepackage{enumerate,color}
\usepackage{url}
\usepackage[pdfborder={0 0 0}]{hyperref}
\usepackage{appendix}
\usepackage{eso-pic}
\usepackage{amsmath}
\usepackage{amssymb}
\usepackage[nottoc]{tocbibind}
\usepackage[sort&compress,authoryear]{natbib}
\usepackage[sf,normalsize]{subfigure}
\usepackage[format=plain,labelformat=simple,labelsep=colon]{caption}
\usepackage{placeins}
\usepackage{tabularx}
\usepackage{wrapfig}
\usepackage{courier}
\usepackage{listings}
\usepackage{tabularx}
\usepackage{booktabs}
% Configurations
\graphicspath{{img/}{../img/}}

\setlength{\parskip}{\baselineskip}
\setlength{\parindent}{0cm}
\raggedbottom
% \setkomafont{subsection}{\normalfont\sffamily}

% Eins og templatið á að vera
% \setkomafont{captionlabel}{\itshape}
% \setkomafont{caption}{\itshape}

% Mun fallegri lausn
\setkomafont{captionlabel}{\itshape}
\setkomafont{caption}{\itshape}
\setkomafont{section}{\FloatBarrier\Large}
\setcapwidth[l]{\textwidth}
\setcapindent{1em}


% Times new roman
%\usepackage[T1]{fontenc}
%\usepackage{mathptmx}

%%%%%%%%%%% MODIFY THESE LINES ONLY %%%%%%%%%%%%%%%%%%%%%%%%%%%%%%%%%%%%%%%%%%%%%%%%%%%%%%%%%
\def\thesisyear{2013}       						% Year thesis submitted
\def\thesismonth{October}						% Month thesis submitted
\def\thesisauthor{Baldur Þór Emilsson}					% Thesis authoreiningaraðferðinni
\def\thesistitle{Marty: Application Development and Testing with Production Data in PostgreSQL} % Title of thesis
\def\thesisshorttitle{Development and Testing with Production Data} 	% Title of thesis
\def\thesiscredits{60} 							% Credits awarded for the project
\def\thesissubject{Computer Science}
\def\thesiskind{M.Sc.}							% Masters of PhD thesis
\def\thesiskindformal{Magister Scientiarum}				% Masters of PhD thesis
\def\thesisnroftutors{1}						% Number of tutors
\def\thesisschool{School of Engineering and {Natural Sciences}}		% School
\def\thesisfaculty{Industrial Engineering,\\Mechanical Engineering and\\Computer Science} % Faculty
\def\thesisaddress{Hjarðarhaga 2-6}					% Office address
\def\thesispostalcode{107, Reykjavik}					% Office address
\def\thesistelephone{525 4700}						% Office telephone
\def\thesistutors{Hjálmtýr Hafsteinsson}
\def\thesisrepresentative{XXNN3}					% Tutors name
\def\thesisPrinting{Háskólaprent, Fálkagata 2, 107 Reykjavík}


%%%% Listing configurations

\lstset{
  aboveskip=2em,
  basicstyle=\footnotesize,
  breaklines=true,
  captionpos=b
}

\renewcommand*{\lstlistlistingname}{List of Listings}

% Function to add footer to frontpage
\newcommand\BackgroundPic{
\put(0,0){
\parbox[b][\paperheight]{\paperwidth}{
\vfill
\centering
\hspace*{-0.6cm}
\includegraphics[width=\paperwidth,height=\paperheight,
keepaspectratio]{foot}
}}
\setlength{\unitlength}{\paperwidth}
\begin{picture}(0,0)(0,-0.15)
\put(0,0){\color{white}\parbox{1\paperwidth}{\centering\bfseries\sffamily \Large Faculty of \thesisfaculty \\
University of Iceland\\
\thesisyear}}
\end{picture}
}

\begin{document}

\begin{titlepage}
\thispagestyle{empty}
\AddToShipoutPicture*{\BackgroundPic}
%
\begin{center}
\vspace*{1cm}
\includegraphics[width=43.6mm]{ui_1_cmyk}\\
\vspace*{3.0cm}
\huge \sffamily \bfseries \thesistitle

\vspace*{5.5cm}
\normalfont \Large \sffamily \thesisauthor
\AddToShipoutPicture*{\BackgroundPic}
\vfill

\end{center}

\newpage 
\thispagestyle{empty} \mbox{}
\newpage
\vspace*{1.35cm}
\thispagestyle{empty}
\begin{center}

\Large \textbf{\sffamily{\MakeUppercase{\thesistitle}}} \\

\vspace*{1.7cm}
\sffamily{\thesisauthor} \\
\vspace*{1.8cm}
\normalsize \thesiscredits~ECTS thesis submitted in partial fulfillment of a \\
\textit{\thesiskindformal} degree in \thesissubject

\vspace*{1.0cm}
\large
\ifnum\thesisnroftutors >1 Advisors \\ \thesistutors \\ \vspace*{0.4cm}
\else Advisor \\ \thesistutors \\ \vspace*{1.04cm}
\fi
Faculty Representative \\
\thesisrepresentative

\vfill
Faculty of \thesisfaculty \\
\thesisschool \\
University of Iceland \\
Reykjavik, \thesismonth~\thesisyear
\newpage
\end{center}
 \newpage
 \thispagestyle{empty}
 \mbox{} \vfill
 % \setcounter{page}{0} \renewcommand{\baselinestretch}{1.5}\normalsize
 \sffamily{\thesistitle} \\
 \sffamily{\thesisshorttitle} \\
 \thesiscredits ~ECTS thesis submitted in partial fulfillment of a \thesiskind~degree in \thesissubject
\\ \\
Copyright \textcopyright~\thesisyear~ \thesisauthor \\
All rights reserved \\


Faculty of \thesisfaculty \\
\thesisschool \\
University of Iceland \\
\thesisaddress \\
\thesispostalcode, Reykjavik \\
Iceland

Telephone: \thesistelephone \\ \\
\vspace*{\lineskip}

Bibliographic information: \\
\thesisauthor, \thesisyear, \thesistitle, \thesiskind~thesis, Faculty of \thesisfaculty, University of Iceland. \\

Printing: \thesisPrinting \\
Reykjavik, Iceland, \thesismonth~\thesisyear \\
\newpage
\end{titlepage}


\pagenumbering{roman}

\setcounter{page}{3}
\section*{\huge Abstract}
Marty is a proof-of-concept prototype for a framework that offers convenient application development and testing against data used in production that is stored in the PostgreSQL database management system.
It is designed for minimal overhead and configuration on production servers while offering quick and simple database initialization on development and testing servers.
This opens the possibility for application testing on production data with minimal effort, which complements conventional testing datasets and helps preventing bugs from entering production code which were not caught with the conventional datasets.
\vfill \vspace*{1cm}
\section*{\huge Útdráttur}
Marty er hugbúnaðarlausn sem býður upp á þægilegt þróunar- og prófunarumhverfi fyrir forrit sem nota PostgreSQL gagnagrunnskerfið.
Hún er hönnuð til að nota gögn úr gagnagrunnum sem keyra í raunumhverfi án þess að hafa neikvæð áhrif á afköst netþjónanna sem grunnarnir keyra á og án mikilla breytinga á uppsetningu þeirra en bjóða á sama tíma upp á fljótlega og einfalda uppsetningu þróunar- og prófunargagnagrunna.
Það opnar fyrir möguleikann á hugbúnaðaprófunum með raungögnum án mikillar fyrirhafnar sem geta keyrt samhliða prófunum með hefðbundin prófunargagnasett og hjálpað við að uppræta villur sem koma ekki í ljós með hefðbundnum prófunum.
\vfill
\newpage

\tableofcontents
\listoffigures
\listoftables
\lstlistoflistings

\chapter*{Abbreviations}
\addcontentsline{toc}{chapter}{Abbreviations}
Í þessum kafla mega koma fram listar yfir skammstafanir og/eða breytuheiti. Gefið kaflanum nafn við hæfi, t.d. Skammstafanir eða Breytuheiti. Þessum kafla má sleppa ef hans er ekki þörf. \\

The section could be titled: Glossary, Variable Names, etc.

\chapter*{Acknowledgments}
\addcontentsline{toc}{chapter}{Acknowledgments}
Í þessum kafla koma fram þakkir til þeirra sem hafa styrkt rannsóknina með fjárframlögum, aðstöðu eða vinnu. T.d. styrktarsjóðir, fyrirtæki, leiðbeinendur, og aðrir aðilar sem hafa á einhvern hátt aðstoðað við gerð verkefnisins, þ.m.t. vinir og fjölskylda ef við á. Þakkir byrja á oddatölusíðu (hægri síðu).

\chapter{Introduction}
\pagenumbering{arabic}
\setcounter{page}{1}
Database management systems (DBMS) are used as data stores in many different systems in various fields.
They are rarely used as standalone products and are typically used to store data from other applications.
These applications are often in constant development with short development cycles, which include both manual and automated testing.
Those  tests are often run against datasets that are created to test for specific conditions and ideally they help with catching all bugs in the applications before they enter production. However, many projects can benefit from tests that are run against data from the production environment, either to complement the testing datasets or to provide data to test against in situations where no testing datasets exist.
The main disadvantage of using production data in testing is that cloning a large database can take a long time which slows down testing and development and it adds an overhead to the database in production which can have negative effects on the performance of the application in the production environment.

The goal of Marty is to offer a convenient and relatively efficient way to run tests for applications that use the PostgreSQL (Postgres) DBMS against live data on the production servers without adding overhead to them.
This is achieved by creating a testing database with empty tables that are populated  when they are first queried.
This saves time as only the tables which are used in the tests are populated and no time is spent copying the data for the other tables, which remain empty.
The data is not copied directly from the production server but from another instance of Postgres that stores a copy of the production data.
This ensures the consistency of the data in the cloned databases and it also minimizes the load on the production database.

The architecture of Marty enables users to inspect the state of the production database as it was at certain points in time in the past.
This is similar to the \textit{time travel} feature which was once a part of Postgres but was removed due to performance and storage space issues.
This can be beneficial in situations where the state of the database caused anomalies or bugs in the application, bugs which have since stopped because the state of the database has changed.
The user could then run the application with data from different points in time to debug it.

\section{Goals and purpose of Marty}
The goal of the development of Marty is to create an application that enables its users to clone a running database quickly.
The original idea was that software developers and testers would be able to clone a database that is used in production and stores large amounts of data that would normally take a considerable time to copy to another server.
Marty will speed up development and testing by reducing the time it takes to clone the production database and also uses techniques that reduce or prevent any negative impact that the cloning would have on the performance of the production database. 

Although the initial idea was for production databases to be cloned there is nothing that prevents Marty from being used for other kinds of databases, such as databases that are dedicated to storing test datasets that never enter production, as long as those databases fulfill the requirements for Marty.

The emphasis in the design and development of Marty is to minimize the time from when the cloning of a database is initialized until the newly created clone can be used for testing.
The performance of the newly created database has not been a high priority as it is not intended to be used in a performance critical environment.
Thus Marty is not a solution that should be used to create clones of a database that are to be used for load balancing or failover or serve any other role in the production environment of an application.

Marty is supposed to be used in an environment where a single or a few databases need to be cloned regularly.
The architecture that was chosen for Marty requires a system administrator to set up and configure Marty for the environment where it is used.
This involves running a dedicated Postgres instance that is used as a reference when the clones are created and also configuring the production server to work with this dedicated instance in a certain way, which might require the production server to be restarted.
It should therefore be clear that Marty is not suited for cloning a database that only needs to be cloned for a limited number of times.
It should be most useful when the database to be cloned is large enough that the time saved by using Marty justifies the initial setup.
\section{Similar solutions}
The development of Marty was started in part to solve a problem that did not have any solutions available.
When a user wanted to replicate a Postgres database she needed to copy the whole database.
Tools like \textit{pg\_dump} exists to aid with database replication but any optimizations had to be created manually for each setup.
If the user wanted to keep some tables empty or skip them completely she needed to create her own solution, such as a script, that implemented that behaviour.
Postgres does not support lazy-loading of data natively like the clone databases in Marty require.

Heroku, a web hosting company, offers Postgres database hosting\footnote{https://www.heroku.com/postgres}.
It has implemented a feature called \textit{forking} which is based on the same idea as Marty but implemented differently.
The documentation for forking includes:

\textit{Preparing a fork can take anywhere from several minutes to several hours, depending on the size of your dataset\footnote{https://devcenter.heroku.com/articles/heroku-postgres-fork}.}

The fork is a complete replica that contains all the tables and database objects of the original database as well as all of its data.
This is unlike the way that Marty creates its clone databases; their tables start empty and are not populated with data until they are queried.
In that sense the forking of Postgres databases in Heroku serves a different purpose than Marty, which puts emphasis on the short initialization time of the clone databases.

There exist numerous clustering solutions for Postgres, such as \textit{Slony-I}\footnote{http://slony.info} and \textit{pgpool-II}\footnote{http://www.pgpool.net/mediawiki/index.php/Main\_Page}.
Many of these solutions can possibly be tailored to suit the needs of developers and testers who need replicas of the production database.
They are, however, not developed with this use case in mind so their usage for this situation can be problematic and can include much configuration and setup, if they can be used at all.
Marty is developed for a specific use case and is tailored to satisfy the requirements of that use case.
This makes it a better choice in the situation where developers and testers need to be able to quickly create replicas of the production database.

Software development includes testing in various stages of the development.
Various methods are used to test the software, such as automated unit tests and integration tests and manual testing that is performed by the developer or a dedicated tester.
It should be possible to use Marty for all stages of software testing.
The creation and initialization of the clone databases should be quick enough to make it feasible to create a new one for each feature that is being developed or tested.
This includes creating a new clone database for each unit test and integration test that is executed, even if there are hundreds of them and they are executed multiple times every day.
\section{Postgres}
Postgres\footnote{http://www.postgresql.org} is an SQL based DBMS that originated at the University of California, Berkeley in the 1980's.
It was based on another DBMS, Ingres, and was released as a free and open source software in 1995.
It is developed by a global community under the name PostgreSQL Global Development Group, with a core team consisting of a half a dozen members and a large number of other contributors.
It is written in C and runs on multiple platforms.

Postgres is very mature and has a large number of features, including conformance with a large part of the SQL standard and a support for extension modules.
Many modules have been created to add new data types, offer new scripting languages for stored procedures and add functionality for specific types of data, such as geographical information.
It has a very extensive documentation and an active community that offers support for users through mailing lists and IRC channels.

Many companies offer commercial support and products based on Postgres with many more using it as a part of their internal systems.
It is used by government organizations and universities and many free and open source projects.

\subsection{Technical information}
There are a few implementation details that are used in this thesis for the discussion of the architecture and implementation of Marty.
Postgres also sometimes uses terminology to describe objects or ideas that is different from the one that is commonly is use.

A \textit{database cluster} in Postgres is a directory in the file system of the server that runs the Postgres instance\footnote{http://www.postgresql.org/docs/9.3/static/creating-cluster.html}.
This directory contains files and subfolders that store all contents of every database that runs on that instance of Postgres.
The files contain binary data that is generally not readable by programs other than the Postgres DBMS.

A \textit{relation} is a database object that stores some data.
One example of a relation is an ordinary table that is created in a Postgres database.
Its contents are stored in the database cluster directory in a \textit{relation file}.
This file consists of \textit{blocks} of data, each of which contains one or many \textit{tuples} that store the values of the relation.
Other types of relations are e.g. views, sequences and foreign tables.

Postgres keeps a log of all changes that are made to the relation files, and other files and directories in the database cluster, in a so-called \textit{write-ahead log} or WAL\footnote{http://www.postgresql.org/docs/9.3/static/wal-intro.html}.
The WAL contains the binary records that Postgres inserts into the files in the cluster.
It is used for recovery after a server crash and can also be used to replicate the database in another instance of Postgres.
Marty uses the WAL to inspect the changes that have been made to the master database which the clone databases replicate.

Postgres is a \textit{transactional} database.
This means that every operation that a user executes in the database is wrapped in a transaction.
Inside a transaction the state of the database is unaffected by the changes that are made in other transactions, even if they run concurrently.
This is important to make it possible for users to query the database even though another user is updating the same tables at the same time.
Postgres implements this by using \textit{multiversion concurrency control} (MVCC)\footnote{http://www.postgresql.org/docs/9.3/static/mvcc.html}.
It enables a database to contain multiple versions of the same data at the same time to ensure that the data in the tables in each transaction is correct and consistent.

The details of how Postgres implements the MVCC are complex, but the most relevant part for the description of Marty are the \textit{xmin} and \textit{xmax} columns.
These are hidden system columns which Postgres adds to every table.
They are used to define which transactions can see which rows in the table.
Every transaction that runs on the server has a \textit{transaction ID}.
The ID of the transaction that inserts a row into a table is recorded in the xmin column.
When a row is deleted the ID of the transaction that deleted it is recorded in the xmax column, until then it contains NULL.
A transaction that updates the values of a row actually inserts a new row with the updated values and marks the old row as deleted.
A row is part of a transaction if the equation \textit{xmin ≤ transaction ID < xmax} holds.

Marty uses a similar technique when it stores multiple versions of the data from the master database.
\section{Thesis Overview}
The rest of the tesis is organized as follows: Chapter \ref{ch:architecture} contains a description of the architecture of Marty and explains the purpose of each part of the system.
Chapter \ref{ch:implementation} contains a detailed description of the implementation of each part, with references to and explanations of the relevant parts of Postgres.
Chapter \ref{ch:current-status} contains the conclusion of the thesis along with a description of the limitations of the current design and ideas for future work.
The source code for the current version of Marty is incuded in Appendix \ref{ch:appendix-source-code}.
Appendix \ref{ch:appendix-name} explains the origins of the name Marty.


\section{Goals and purpose of Marty}
The goal of the development of Marty is to create an application that enables its users to clone a running database quickly.
The original idea was that software developers and testers would be able to clone a database that is used in production and stores large amounts of data that would normally take a considerable time to copy to another server.
Marty will speed up development and testing by reducing the time it takes to clone the production database and also uses techniques that reduce or prevent any negative impact that the cloning would have on the performance of the production database. 

Although the initial idea was for production databases to be cloned there is nothing that prevents Marty from being used for other kinds of databases, such as databases that are dedicated to storing test datasets that never enter production, as long as those databases fulfill the requirements for Marty.

The emphasis in the design and development of Marty is to minimize the time from when the cloning of a database is initialized until the newly created clone can be used for testing.
The performance of the newly created database has not been a high priority as it is not intended to be used in a performance critical environment.
Thus Marty is not a solution that should be used to create clones of a database that are to be used for load balancing or failover or serve any other role in the production environment of an application.

Marty is supposed to be used in an environment where a single or a few databases need to be cloned regularly.
The architecture that was chosen for Marty requires a system administrator to set up and configure Marty for the environment where it is used.
This involves running a dedicated Postgres instance that is used as a reference when the clones are created and also configuring the production server to work with this dedicated instance in a certain way, which might require the production server to be restarted.
It should therefore be clear that Marty is not suited for cloning a database that only needs to be cloned for a limited number of times.
It should be most useful when the database to be cloned is large enough that the time saved by using Marty justifies the initial setup.
\chapter{Architecture}
This chapter contains a detailed description of the architecture and design of Marty.

\section{Overview}
Marty consists of a few parts that serve different purposes, see figure \ref{architecture-overview}.
Developers and testers that use Marty in their work create database \textit{clones}.
These databases are clones of the \textit{master} database, which can be a database that is used in a production environment.
When the clones are created they are initialized with a copy of all the tables in the master, but the tables remain empty until they are queried by a user or an application.
The design of the clone databases is discuessed in chapter \ref{sec:clones}.

\begin{figure}[h!]
  \centering
    \includegraphics[width=0.8\textwidth]{architecture}
  \caption{An overview of the architecture of Marty}
  \label{architecture-overview}
\end{figure}

Marty does not inspect the schema of the master database directly when it creates the tabes in the clone databases.
Instead it queries another database that is called \textit{history}.
The history database contains information about the schema of the master database as well as a copy of its data.
When the clones need to populate their tables they also use this history database as a reference.
The reason for using another database to store a copy of the schema and data of the master database is discussed in chapter TODO, along with a description of the design of the history database.

As the name suggests the history database contains not only a copy of the current version of the master but also of previous versions.
To update the history database with new versions of the master and to keep it in sync with the changes that are made on the master Marty uses a log from the master that is called the \textit{write-ahead log} or \textit{WAL}.
It does not read this log directly but uses a specially patched instance of Postgres to read the contents of the log.
This instance is called \textit{slave} and it outputs information that Marty can use to update the history database with all the changes that have been made in the master database.
The reason for keeping old versions of the master database and the relationship between the master, slave and history databases is described in detail in chapter TODO.

\section{The clone databases}
\label{sec:clones}
A clone database is a standard Postgres database.
It uses two Postgres extensions; the \textit{PL/pgSQL} extension that enables users to create stored procedures, and the \textit{dblink} extension which enables users to query another database directly from the clone database without using any external scripts or programs.
The clone can run on a local instance of Postgres on the developers' or testers' computer as long as the history database is accessible from that computer.
More than one clone database can run in parallel on the same instance of Postgres so each user can use many clones at the same time.

To create a clone the user creates a new, empty database.
She then initializes it with Marty.
After the clone has been initialized it contains all the schemas that are found in the master database and a copy of all the tables from each schema.
The tables remain empty until they are first queried which saves time in the initialization as the user does not have to wait for Marty to finish copying all the data in the tables before she can start querying the clone.
This behaviour is implemented by creating views instead of tables in the clone.
The views look like the tables that the user expects to find and when they are queried they call a PL/pgSQL function, \textit{view\_select}, that returns the appropriate data.
This function looks for the data in the actual data tables, which Marty creates in the clone, and if these tables are empty the function populates them with data from the history database before returning their contents.

The data tables are created in a special schema called \textit{marty} which is created in the clone database.
It contains the data tables as well as another table called bookkeeping.
The view\_select function uses this table to keep track of which data tables have been populated and which ones are still unpopulated.
The table also contains the querystrings that view\_select uses when it fetches the data from the history database.
See figure \ref{clone-architecture} for an example of a table layout in a clone database.

The views, bookkeeping table, data tables and the view\_select function are described in detail in chapter TODO.

\begin{figure}[h!]
  \centering
    \includegraphics[width=0.5\textwidth]{clone-architecture}
  \caption{Table layout in a clone database}
  \medskip
  \small
  The table \textit{mytable} is actually a view that returns results from the table \textit{data\_myschema\_mytable\_1} which is in the \textit{marty} schema.
  \label{clone-architecture}
\end{figure}

\section{The history database}
The history database is a standard Postgres database.
It is created by a system administrator and contains data that the developers and testers use when they create clone databases.
When the history database has been initialized it contains information about the schema of the master database and a copy of its data.
After the initialization Marty updates the history database with all the changes that are made in the master, both to its data and schema.
Its contents are versioned and, as the name suggests, the user can look up previous states of the master in the history database.

The reason for keeping old versions of the master is the delayed population of the data tables in the clones.
From the moment that a clone database is initialized and until its tables are populated the master database might change.
Tables might be dropped or renamed and rows might be updated or deleted, which could lead to inconsistency in the clone as foreign key relations might break.
The history database offers access to a particular version of the contents of the master database and thus prevents errors of this kind in the clones.

Marty initializes the history database with four tables; \textit{marty\_schemas}, \textit{marty\_tables}, \textit{marty\_columns} and \textit{marty\_updates}.
The first three store information about the schemas, tables and table columns in the master database.
Their contents are mostly copied from the tables \textit{pg\_namespace}, \textit{pg\_class} and \textit{pg\_attributes}, respectively.
The fourth table, \textit{marty\_updates}, keeps a log of version timestamps.
Each version of the contents of the history database has two timestamps associated with it; the local time of the history server when that version was created in the history database and the time of the transaction on the master database that created that version.

Marty copies the data from the tables in the master and stores it in special data tables in the history database.
Each table in the master has a corresponding data table in the history database.
Its schema is similar to the original table but a few columns are added.
They are used for versioning and as a reference when rows are deleted or updated.
There are also no constraints on the data tables or their columns.
They are unnecessary as the data table is only used to store the data that has already been validated on the master.
Constraints might also get in the way, e.g. when the tables need to store different versions of the same row that has a unique constraint on some of its columns.
Another example might be a table that is altered and a not-null constraint is added to one of its columns where null values have been stored in the past.
Therefore there are no constraints on the data tables in the history database.
For an example of the schema of a data table see figure \ref{history-data-table}, and for further details see chapter TODO.

\begin{figure}[h!]
  \centering
    \includegraphics[width=0.5\textwidth]{history-data-table}
  \caption{An example of the schema of a data table in the history database}
  \label{history-data-table}
\end{figure}

\subsection{Populating and updating the history database}
When the history database is initialized its contents are not read directly from the master database.
Instead there is a dedicated instance of Postgres that replicates the master database that is used as a reference for the history.
This instance is called \textit{slave}.
The reason for using another instance to read the schema and data from the master is the format of the write-ahead log, or WAL.

The contents of the WAL are used to update the history database with the changes that are made to the master after the history has been initialized.
This approach was chosen because it makes it possible to read the changes that are made to the master database without inspecting it directy, e.g. with triggers.
That was considered important in the design process for Marty because any change on the master might introduce bugs and reduce its performance.
However, the WAL contains binary information and to be able to read it and use its contents it is necessary to have the database cluster files from the master as a reference.
Instead of implementing a complex algorithm to read the contents of the WAL it was decided to leverage on the recovery feature of Postgres that reads the WAL and applies it to the database cluster.
The slave is therefor started with a copy of the cluster files from the master database and it then replays the WAL into the cluster as it arrives from the master.
As it does so it logs all the operations that it replays and Marty uses this log to find and read the new version of the data from the slave.

The WAL debug log that logs all the operations is not enabled by default and the slave must therefor be compiled with special flags to enable it.
The replay must also be paused after each transaction that has been replayed to give Marty time to read the data from that transaction before the next one is applied.
To be able to do that it is necessary to patch Postgres and so the slave therefor runs on a specially patched version of Postgres.
More information about the slave can be found in chapter TODO.

\section{Advantages and Drawbacks}
The current architecture of Marty that is described in this chapter was chosen because of its simplicity and because it could be implemented in high level code (PL/pgSQL instead of C) which sped up prototyping and simplified the development of Marty.
However, it has a few drawbacks which make it unsuitable for a production ready version of Marty.
The main drawback is the lack of optimization for queries from the clones to the history database; when a user queries a table in a clone database it fetches the complete contents of that table from the history database even if the query should only return a small part of it to the user.
Another issue is the creation of indexes for the tables in the clones; the user can not create indexes for the tables in the clone like she would create them on the master database.
This is because the tables that the user expects to find in the clone database are actually views and it is not possible to create indexes for views in Postgres.

See chapter TODO for a discussion of the current status of Marty, the limitations of the current version and ideas for future works and improvements.

\chapter{Implementation}
Marty is written in Python and PL/pgSQL with a small patch to the Postgres source code written in C.
Python and PL/pgSQL are both high-level programming languages and ideal for rapid prototyping.
The source code for Marty contains two scripts, \textit{clone.py} and \textit{history.py}, which are used to create and populate the clone databases and the history database, respectively.
The patch to Postgres is necessary for Marty to be able to read the changes from the write-ahead log (WAL) with the slave instance.
It is for version 9.3.3 of Postgres and might not work with other versions.

This chapter describes the implementation of Marty.
It explains which parts of Postgres Marty uses to create the history database and keep track of the changes that are made to the master database.
It starts by explaining how the slave instance is used and why it is patched.
Next it describes the history database and its design and then continues with a description of the clone databases and how they use the history database.
The last part of this chapter describes briefly how the source code for Marty is organized.

\section{The slave instance}

\begin{wrapfigure}{r}{0.4\textwidth}
  \vspace{-20pt}
  \begin{center}
    \includegraphics[width=0.38\textwidth]{img/architecture-slave}
  \end{center}
  \vspace{-20pt}
  \caption{The slave part of the architecture}
  \vspace{-10pt}
\end{wrapfigure}

Marty uses the slave instance to initialize the history database and to inspect the contents of the write-ahead log from the master.
The slave is configured to act as a \textit{hot standby} for the master; it starts with a copy of the master database and updates it with the WAL.
When the slave database is first started Marty copies its schema and data to the history database.
It then inspects the changes from the WAL as they are applied and updates the history database accordingly.

Before the slave instance is started a database administrator must configure the master.
This includes configuring a few parameters in the \textit{postgres.conf} file, see table \ref{tbl:master-config}.
Next the administrator must create a \textit{base backup} of the master database.
A base backup is a copy of the cluster files that store all the data in the database.
It can be created with the program \textit{pg\_basebackup} and might require changes to the \textit{pg\_hba.conf} file, see the Postgres documentation for further reference. % TODO add referernce?

\begin{table}[h]
  \centering
  \texttt{
    \begin{tabular}{| l | l |}
      \hline
      \textbf{Parameter} & \textbf{Value} \\ \hline
      wal\_level & hot\_standby \\ \hline
      archive\_mode & on \\ \hline
      archive\_command & 'cp \%p /path/to/archive/\%f' \\ \hline
      max\_wal\_senders & 1 \\ \hline
    \end{tabular}
  }
  \caption{Configuration parameters in postgres.conf for the master database}
  \medskip
  \small
  The archive command is an example, when the master is configured it must use an archive command that copies the WAL files to a storage where they are accessible by the slave.
  Also note that \texttt{max\_wal\_senders} must at least be 1, but can be higher.
  \label{tbl:master-config}
\end{table}

As previously noted the slave runs on a patched version of Postgres that must be compiled with a special flag that enables Postgres to log the WAL replay actions.
When the patch has been applied to the Postgres source code it must be compiled with the \texttt{WAL\_DEBUG} CPP flag.

Instead of creating a new database cluster for the slave with the \texttt{initdb} command the administrator uses the base backup from the master.
When it has been copied to the correct place the \textit{postgres.conf} file must be updated, see table \ref{tbl:slave-config}.
It is then necessary to add a \textit{recovery.conf} file with a command to fetch the WAL files from the master database, see the Postgres documentation for further reference. % TODO reference?

\begin{table}[h]
  \centering
  \texttt{
    \begin{tabular}{| l | l |}
      \hline
      \textbf{Parameter} & \textbf{Value} \\ \hline
      hot\_standby & on \\ \hline
      wal\_debug & on \\ \hline
    \end{tabular}
  }
  \caption{Configuration parameters in postgres.conf for the slave database}
  \label{tbl:slave-config}
\end{table}

\subsection{Reading the WAL}
\label{ch:reading-the-wal}
The write-ahead log contains all the changes that are applied to the master database.
They are stored in binary records that can be applied directly to the files in the slave database cluster to repeat the changes on the slave.
There are a few different kinds of WAL records that store information about different operations on the master.
The ones that Marty looks for are \textit{heap} records and \textit{commit} records.

Heap records contain changes to the tables in the database; inserts, updates and deletes.
The commit records signal the slave to commit and close the current transaction that is being replayed from the WAL.
Other kinds of WAL records include B-tree records for B-tree indexes and XLOG records that contain information about transaction logs.

Marty can read information about the heap records in the WAL replay log.
The log contains information such as the type of the heap record; insert, update or delete, and identifiers of the database and table that the record alters.
The log also contains references to the row which was inserted, updated or deleted.

\begin{lstlisting}[caption={WAL replay log example},label={lst:wal-replay-log},numbers=left,xleftmargin=2em]
LOG:  REDO @ 0/800F1A0; LSN 0/800F248: prev 0/800F160; xid 741; len 139: Heap - insert: rel 1663/16384/11829; tid 39/63
LOG:  REDO @ 0/900390C; LSN 0/9003944: prev 0/9002158; xid 742; len 26: Heap - delete: rel 1663/16384/11829; tid 39/38 KEYS_UPDATED
LOG:  REDO @ 0/8011D80; LSN 0/8011F0C: prev 0/8011D54; xid 741; len 368: Transaction - commit: 2014-03-06 23:36:33.937958+00; inval msgs: catcache 59 catcache 58 catcache 59 catcache 58 catcache 45 catcache 44 catcache 7 catcache 6 catcache 7 catcache 6 catcache 7 catcache 6 catcache 7 catcache 6 catcache 7 catcache 6 catcache 7 catcache 6 catcache 7 catcache 6 relcache 16394
\end{lstlisting}

Listing \ref{lst:wal-replay-log} has an example of the WAL replay log.
The insert and delete heap records in lines 1 and 2 in this example alter the same table.
This table can be looked up with the \textit{rel} values in the log; 1663 is the database ID, 16384 is the namespace ID and 11829 is the table ID.
Marty uses the database and table IDs to look up the table in the slave database and queries it with the \textit{tid} values, which in this case are 39/63 and 39/38.
The tid values reference the row which was inserted to or deleted from the table.
The last line of the example, line 3, logs a commit record.
It closes the transaction and applies the changes from lines 1 and 2 to the slave database.

The next section describes the schema of the history database and how Marty queries the slave database for data to insert into the history.
\section{The history database}

\begin{wrapfigure}{r}{0.4\textwidth}
  \vspace{-20pt}
  \begin{center}
    \includegraphics[width=0.38\textwidth]{img/architecture-history}
  \end{center}
  \vspace{-20pt}
  \caption{The history part of the architecture}
  \vspace{-10pt}
\end{wrapfigure}

The history database contains information about the schema of the master database and a copy of its data.
It contains multiple versions of the schema and data from the master, each transaction that alters the master database creates a history version.
It is used as a reference when the user creates a new clone database; the schema of the clone is created according to the information in the history database and its data is copied from the history.

The history database is created when the slave instance has been configured correctly.
It is a standard Postgres database that must have the same minor version as the master database.
That means that if the master runs on Postgres version 9.3.3 the history version must start with 9.3, see the Postgres documentation for further reference. % TODO add reference?

\subsection{Schema information}
To initialize the history database the administrator runs the script \textit{history.py}.
It starts by creating the \textit{schema information tables} which Marty uses to store information about the schema of the master database:

\begin{description}
  \item[marty\_schemas]
    Contains the name and ID of all schemas in the slave database.
  \item[marty\_tables]
    Contains the name and ID of the tables in the slave database and a reference to the schema they are in.
    It also stores the \textit{internal name} which is used for the data table in the history and clone databases.
  \item[marty\_columns]
    Contains the name, number, type and length of each column in the tables.
    It also stores an internal name for the column that is used in the data tables in the history and clone databases.
\end{description}

Tables \ref{table:marty-schemas} to \ref{table:marty-columns} show information about the columns of these tables.

\begin{table}[h]
  \centering
  \textbf{marty\_schemas}
  \begin{tabularx}{\textwidth}{llX}
    \textit{Column} & \textit{Type} & \textit{Description} \\
    \midrule
    \_ctid & tid & A reference to the pg\_namespace table in the slave database \\
    oid & oid & The ID of the schema in the slave database \\
    name & name & The name of the schema \\
    start & integer & First version where this schema is present in the database \\
    stop & integer & First version where this schema stops being present in the database \\
  \end{tabularx}
  \caption{The columns of the marty\_schemas table}
  \label{table:marty-schemas}
\end{table}

\begin{table}[h]
  \centering
  \textbf{marty\_tables}
  \begin{tabularx}{\textwidth}{llX}
    \textit{Column} & \textit{Type} & \textit{Description} \\
    \midrule
    \_ctid & tid & A reference to the pg\_class table in the slave database \\
    oid & oid & The ID of the table in the slave database \\
    name & name & The name of the table \\
    schema\_oid & oid & A reference to the schema that this table belongs to \\
    internal\_name & name & The name of the data tables in the history and clone databases \\
    start & integer & First version where this table is present in the database \\
    stop & integer & First version where this table stops being present in the database \\
  \end{tabularx}
  \caption{The columns of the marty\_tables table}
  \label{table:marty-tables}
\end{table}

\begin{table}[h]
  \centering
  \textbf{marty\_columns}
  \begin{tabularx}{\textwidth}{llX}
    \textit{Column} & \textit{Type} & \textit{Description} \\
    \midrule
    \_ctid & tid & A reference to the pg\_attribute table in the slave database \\
    table\_oid & oid & A reference to the table this column is in \\
    name & name & The name of the column \\
    number & int2 & The index of this column in the table (is it the first, second, third etc.) \\
    type & name & The type of the column (int, text, boolean etc.) \\
    internal\_name & name & The name of this column in the data tables in the history and clone databases \\
    start & integer & First version where this column is part of the table \\
    stop & integer & First version where this column stops being part of the table \\
  \end{tabularx}
  \caption{The columns of the marty\_tables table}
  \label{table:marty-columns}
\end{table}

Postgres uses \textit{system catalogs} to store information about the schema of its databases.
They are used internally, e.g. when Postgres reads from or writes to tables.
Things such as which file in the database cluster represents which table and the column names and types of each table are stored in the system catalogs. % TODO Don't use Things such as...
They are ordinary relations which are stored in a schema called \textit{pg\_catalogs} and can by queried by a user just as any other relation.

Marty reads information from four system catalogs in the slave database:

\begin{description}
  \item[pg\_namespace]
    Contains information about the schemas in the slave database.
    Information from this catalog is saved in \textit{marty\_schemas} in the history database. % TODO saved in or written to?
  \item[pg\_class]
    Contains information about the tables in the slave database.
    Information from this catalog is saved in \textit{marty\_tables} in the history database.
  \item[pg\_attribute]
    Contains information about the columns of the tables in the slave database.
    Information from this catalog is saved in \textit{marty\_columns} in the history database.
  \item[pg\_type]
    Contains additional information about the columns.
    Marty reads the name of the column type from this table (integer, text etc.) and stores it in \textit{marty\_columns} along with the other column related information.
\end{description}

The history.py script creates the schema information tables and then it populates them with information about the schema of the slave database.
This is of course also the schema of the master database so the history database really contains information about the master.

Marty inspects the schema of the slave database before any WAL records have been applied to it.
It writes the schema information to the history database, which creates the first history version.
Marty then starts the WAL replay in the slave database and reads the replay log.

When a new schema is created the WAL contains an \textit{insert} heap record for the pg\_namespace table.
Marty sees this in the replay log and queries the slave about this new schema.
It then creates a new history version that includes it.
The same happens when a new table is created or a new column is added to a table; a new history version is created that includes the new table or column.

If a schema, table or column is altered, e.g. when they are renamed, the WAL has an \textit{update} heap record.
Similarly when a schema, table or column is dropped the WAL has a \textit{delete} heap record.
Marty sees this in the replay log and creates new history versions with the altered database schema.
\section{The clone databases}

\begin{wrapfigure}{r}{0.4\textwidth}
  \vspace{-20pt}
  \begin{center}
    \includegraphics[width=0.38\textwidth]{img/architecture-clones}
  \end{center}
  \vspace{-20pt}
  \caption{The clone part of the architecture}
  \vspace{-10pt}
\end{wrapfigure}

A clone database is a replica of the master database.
The user can query the clone just like she would query the master.
It is a standard Postgres database that uses the \textit{PL/pgSQL} and \textit{dblink} extensions.
The user creates a new, empty database and runs the script \textit{clone.py} to initialize it as a clone.
The script reads the schema information from the history database and initializes the clone accordingly.

Like the history database the clone database must have the same minor version as the master.
If the master has the version 9.3.3 that means that the clone databases must have a version number that starts with 9.3.

The initialization script creates schemas in the clone database to match those in the master database.
The tables in the clone are lazy loading.
This means that a table is empty until a user queries it for the first time.
The clone then fetches its content from the history database.
This is implemented with a view which the user queries and an accompanying data table which stores the data once it has been fetched from the history.

The data tables live in a schema called \textit{marty}.
They have the same name as the data tables in the history database (see chapter \ref{ch:implementation-history-data}).
The columns are identical to the columns in the original table in the master database, there are no extra columns and their names are identical to the column names in the original table.
See figure \ref{fig:clone-tables-2} for an example of a data table.

\begin{figure}[h!]
  \centering
    \includegraphics[width=0.6\textwidth]{clone-tables}
  \caption{Table layout in a clone database}
  \medskip
  \small
  The table \textit{persons} is actually a view that returns results from the table \textit{data\_myschema\_persons\_1} which is in the \textit{marty} schema.
  \label{fig:clone-tables-2}
\end{figure}

When the user queries a table in the clone database like she would query it in the master she is actually querying the view that Marty creates.
It executes a PL/pgSQL function, \textit{view\_select}, which returns the contents of the original table in the master.
The function is created by Marty when the clone is initialized.
It receives one argument, the name of the view that is being queried, and fetches the data for it from the corresponding data table.
When the view is queried for the first time the view\_select function fetches the data from the history database and inserts it into the data table before returning the results.
See listing \ref{lst:view-select} for the source code of the view\_select function.

\lstset{
  language=SQL,
  morekeywords={
    FUNCTION, RETURNS, SETOF, RECORD, DECLARE, BEGIN,
    IF, RAISE, NOTICE, RETURN, QUERY, LANGUAGE
  }
}
\begin{lstlisting}[caption={The view\_select function},label={lst:view-select},numbers=left,xleftmargin=2em]
CREATE FUNCTION view_select(my_view_name text) RETURNS SETOF RECORD AS $$
DECLARE
  view_info RECORD;
BEGIN
  SELECT * FROM marty.bookkeeping WHERE view_name = my_view_name INTO view_info;
  IF NOT view_info.cached THEN
    RAISE NOTICE 'fetching %', view_info.view_name;
    EXECUTE ' INSERT INTO ' || view_info.local_table ||
      ' SELECT ' || view_info.coldef ||
      ' FROM dblink(''' || coninfo() || ''', ''' || view_info.remote_select_stmt || ''')'
      ' AS ' || view_info.temp_table_def;
	UPDATE marty.bookkeeping SET cached = true WHERE view_name = my_view_name;
  END IF;
  RETURN QUERY EXECUTE 'SELECT ' || view_info.coldef || ' FROM ' || view_info.local_table;
END;
$$ LANGUAGE plpgsql;
\end{lstlisting}

Marty keeps track of which data tables have been populated in the table \textit{bookkeeping}.
It is created alongside the data tables in the marty schema when the clone database is initialized and is used to store information about the data tables and views.
It stores which data table keeps the data for which view and whether it has been initialized, as well as information about how to query the history database to fetch the data for each view.
See table \ref{tbl:bookkeeping} for a list of columns in the bookkeeping table.

\begin{table}[h]
  \centering
  \textbf{bookkeeping}
  \begin{tabularx}{\textwidth}{llX}
    \textit{Column} & \textit{Type} & \textit{Description} \\
    \midrule
    view\_name & name & The name of the view \\
    local\_table & name & The name of the data table that contains the data for the view \\
    cached & boolean & True if the data table has been populated with data from the history database \\
    coldef & text & A list of columns in the data table, used by the \textit{view\_select} function when querying the data table \\
    remote\_select\_stmt & text & The select statement for the data table in the history database \\
    temp\_table\_def & text & Table definition for a temporary table, used by view\_select \\
  \end{tabularx}
  \caption{The columns of the bookkeeping table}
  \label{tbl:bookkeeping}
\end{table}

The \textit{cached} column is a boolean which tells the view\_select function whether the data table has been populated with data from the history database (line 6 in listing \ref{lst:view-select}).
It defaults to false as all data tables start empty.
The function uses the \textit{coldef} value when it queries the data table.
It is a comma separated list of the columns in the data table and is used to construct the select query (line 14).
When the view\_select function queries the history database it uses the \textit{remote\_select\_stmt} as the query (lines 7 to 11).
It is a select query that returns all rows from the data table in the history.
The \textit{dblink} function from the dblink extension is used to query the history database (line 10).
It receives two parameters; a connection string and the query to run.

The connection string is generated when the clone database is initialized.
Marty creates a function, \textit{coninfo}, that returns this string.
This makes it trivial to update the connection string later in a running clone database if the setup of the history database changes, as only this one-line function needs to be redefined instead of the whole view\_select function.
The query that returns the data from the history database is created when Marty initializes the clone and is saved in the \textit{remote\_select\_stmt} column in the bookeeping table.

The dblink query must have an alias part where the names and types of the columns in the result rows are specified.
An example of this could be \texttt{SELECT * FROM dblink('dbname=mydb', 'SELECT age, name FROM persons') AS p(age int, name text)}.
The part after \textit{AS} tells Postgres what to expect in the query results.
This part is saved in the \textit{temp\_table\_def} column in the bookkeeping table and is appended to the dblink query in the view\_select function (line 11).
\section{Source code}
The source code for Marty contains two Python scripts that can be run from the command line.
Those are \textit{history.py}, which a database administrator uses to initialize the history database, and \textit{clone.py} which a user runs to initialize a clone database.
The rest of the code consists of three Python files which contain classes that the scripts use.

\begin{description}
  \item[inspector.py]
    Contains the classes \textit{SlaveInspector} and \textit{HistoryInspector}.
    The SlaveInspector is used by history.py to gather information about the schemas and tables in the slave database.
    The HistoryInspector is used by clone.py to read the schema and table information from the history database.
  \item[populator.py]
    Contains the classes \textit{HistoryPopulator} and \textit{ClonePopulator}.
    The HistoryPopulator is used by history.py to insert the information from the SlaveInspector into the history database.
    The ClonePopulator is used by clone.py to create schemas and tables in the clone database according to the information from the HistoryInspector.
  \item[dbobjects.py]
    This file contains a few classes that the inspectors use to supply the populators with data.
    They include \textit{Schema}, \textit{Table} and \textit{Column} which contain the name and other data about each one of those objects.
    The Table and Column classes also have methods to create the names of the data tables and its columns.
\end{description}

The history.py script receives ten arguments:

\texttt{---slave-host} The hostname or IP address of the slave database \\
\texttt{---slave-port} The port number of the slave database \\
\texttt{---slave-user} The username for the slave database \\
\texttt{---slave-password} The password for the slave database \\
\texttt{---slave-database} The name of the slave database \\
\texttt{---history-host} The hostname or IP address of the history database \\
\texttt{---history-port} The port number of the history database \\
\texttt{---history-user} The username for the history database \\
\texttt{---history-password} The password for the history database \\
\texttt{---history-database} The name of the history database

The clone.py script receives the same arguments for the history database and similar arguments for the clone database that should be initialized.

The clone.py script only uses the data in the history database to populate the clone.
The history.py script receives the WAL replay log from the slave database as standard input.
It uses a couple of classes to consume and parse the log, \textit{Worker} and \textit{RegExer}.
The Worker class reads the log and keeps a list of all heap records it encounters.
When it reads a commit record it creates a new history version in the history database and runs through the heap records, saving the changes to the new history version.
To parse the log records it uses the RegExer class, which is a wrapper around regular expressions which are used to match and extract data from the replay log.
See chapter \ref{ch:reading-the-wal} for an explanation of the WAL replay log, heap records and commit records.

All SQL and PL/pgSQL code in Marty is part of the Python scripts.
Marty is not a large project and the current version contains a little under 1100 lines of code.
Included in that number is the file that contains the patch to the Postgres source code.

% TODO add a reference to the souce code in the Appendix
\chapter{Current status and future work}
\label{ch:current-status}
The current version of Marty is a proof of concept prototype.
It supports replicating a database with ordinary tables without any default values or constraints.
It does not support other types of relations or other kinds of database objects other than schemas.

\section{Production use}
Before Marty is ready for use in a production environment support must be added for at least default values on columns, column and table constraints, sequences and indexes.
The last two are implemented as relations in Postgres, other types of relations are views and materialized views, TOAST tables, composite types and foreign tables.
Support for these types would be nice to have, although the data in foreign tables is not a part of the database and is not logged in the write ahead log.
This means that the history database can not save the contents of foreign tables and support for them would need to be tailored for each setup of Marty.

Other types of database objects that could be supported in future versions are functions and operators, data types and domains, triggers and rewrite rules.

The clone databases lazy-load the contents of their tables by using views instead of tables.
This prevents many operations that a user might want to perform in the database, such as creating new indexes or altering the tables.
To create an index it must be added to the data tables in the marty schema, and when the user wants to alter a table she must alter the data table as well as the view.
Future versions of Marty could include a patched version of Postgres for the clone databases.
It would include a relation type that looked and behaved like an ordinary table but would lazy-load its data from the history database behind the scenes.

Using a patched version of Postgres opens the possibility for optimizing the queries that are run on the history database.
Currently when a user queries a table in a clone database it fetches the complete contents of that table from the history database.
This is unncecessary when query returns only a part of the data from the table.
Optimized queries would reduce the time that queries take to return from the history database, which in turn would make the clone database return results faster to the user.
Marty could optimize queries that contain a \textit{WHERE} clause or a \textit{LIMIT} clause, and also queries that aggregate the data in some way, such as queries using the \textit{SUM}, \textit{MIN} or \textit{MAX} functions.

\section{Logical replication}
\textit{Logical log streaming replication} is a feature that is being implemented for Postgres and will be part of future versions.
It provides the possibility of shipping the change log of one database instead of shipping the changes of the whole database cluster like the write ahead log currently does.
The receiving, or slave, database reads the contents of the logical log with plugins.
One plugin is used to apply the changes to the slave database but another one is available that creates SQL statements with the changes, without applying them.
Marty could provide one such plugin for the history database that would read the log and create new history versions.
This would make the slave instance in Marty unnecessary and would simplify the setup and maintenance of Marty.

\section{Data obfuscation}
Many databases contain sensitive information, or information that should not be used in a development or testing environment.
Examples of such data are credit card numbers and e-mail addresses.
Marty could include the possibility to change the values of certain columns in certain tables when it inserts data into the history database.
This could be in the form of a Python script which would allow for a very flexible way of obfuscating or changing the values when they are read from the slave database.
This would prevent the sensitive data from entering the development or testing environment, thus preventing any accidental use.

\section{VCS integration}
Version control systems (VCS) such as Git and Mercurial allow developers to work on many different features or fixes for a program simultaneously while keeping the changes for each feature isolated.
The developers create \textit{branches} of the source code where each branch contains the changes for only one feature or fix.
They can then switch between branches without mixing the changes from one feature with the changes from another one.

% TODO add references to git and mercurial?
It is possible to create \textit{hooks} for these systems that run specific commands when the user executes certain actions.
Marty could include hooks that would automatically create and initialize a new clone database whenever a developer created a new branch.
They could also be configured to updated the settings of the project that the developer is working on.
The developer would then always use the correct database for the active branch without needing to change manually from one database to another.
This could ease development as the developer would always get a new development database for each new branch.

\section{History inspection}
The history database contains information about when and how certain changes were made on the master database.
This information could be useful in certain situations where it would be beneficial to inspect the state of the master at a given point in time.
This is already possible by using the write ahead log, but it is not a very practical solution.
The WAL must be applied to an old base-backup of the master database and Postgres must be configured to stop the WAL replay at the right moment.
This can be time consuming as the WAL replay can take considerable time.
It can also be hard to leap between different points in time when replaying the WAL as there is no way to rewind it.

The information in the history database can be used to quickly inspect the state of the master as it was at a certain point in time.
It would be a relatively quick operation to jump backwards or forwards in time and inspect the different states of the database.
It is already possible to use Marty in such a way with minimal changes.
The user selects which version to inspect and creates a new clone database that is initialized for that version.
This includes some manual labor as the use must find the correct version ID and must configure Marty to initialize a clone database with that specific history version.
It is also cumbersome to inspect different versions as the user would need to create a new clone database for each version and then query each one and inspect the results manually.

Future versions of Marty could include a tool that would enable a user to quickly scan through and locate the correct history version.
It would allow the user to inspect the database as it was at that time and even compare two or more versions.
This would make it possible to debug anomalies that were caused by master database even after the database state has been altered and the anomalies have stopped.
\include{conclusion}

\appendix
\renewcommand{\chaptername}{Appendix}

\chapter{Source code}
This is the source code for Marty.
It is accessible online at Github\footnote{https://github.com/baldurthoremilsson/marty/commit/f9a88d7}.

\definecolor{deepblue}{rgb}{0,0,0.5}
\definecolor{deepgreen}{rgb}{0,0.5,0}

\lstset{
  basicstyle=\tiny\ttfamily,
  breaklines=true,
  captionpos=t,
  language=Python,
  numbers=left,
  xleftmargin=2em,
  keywordstyle=\bfseries\color{deepblue},
  stringstyle=\color{deepgreen},
}

\input{source-code-clone}
\input{source-code-history}
\input{source-code-init}
\input{source-code-dbobjects}
\input{source-code-inspector}
\input{source-code-populator}
\input{source-code-patch}


\end{document}
