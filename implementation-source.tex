\section{Source code}
The source code for Marty contains two Python scripts that can be run from the command line.
Those are \textit{history.py}, which a database administrator uses to initialize the history database, and \textit{clone.py} which a user runs to initialize a clone database.
The rest of the code consists of three Python files which contain classes that the scripts use.

\begin{description}
  \item[inspector.py]
    Contains the classes \textit{SlaveInspector} and \textit{HistoryInspector}.
    The SlaveInspector is used by history.py to gather information about the schemas and tables in the slave database.
    The HistoryInspector is used by clone.py to read the schema and table information from the history database.
  \item[populator.py]
    Contains the classes \textit{HistoryPopulator} and \textit{ClonePopulator}.
    The HistoryPopulator is used by history.py to insert the information from the SlaveInspector into the history database.
    The ClonePopulator is used by clone.py to create schemas and tables in the clone database according to the information from the HistoryInspector.
  \item[dbobjects.py]
    This file contains a few classes that the inspectors use to supply the populators with data.
    They include \textit{Schema}, \textit{Table} and \textit{Column} which contain the name and other data about each one of those objects.
    The Table and Column classes also have methods to create the names of the data tables and its columns.
\end{description}

The history.py script receives ten arguments:

\texttt{---slave-host} The hostname or IP address of the slave database \\
\texttt{---slave-port} The port number of the slave database \\
\texttt{---slave-user} The username for the slave database \\
\texttt{---slave-password} The password for the slave database \\
\texttt{---slave-database} The name of the slave database \\
\texttt{---history-host} The hostname or IP address of the history database \\
\texttt{---history-port} The port number of the history database \\
\texttt{---history-user} The username for the history database \\
\texttt{---history-password} The password for the history database \\
\texttt{---history-database} The name of the history database

The clone.py script receives the same arguments for the history database and similar arguments for the clone database that should be initialized.

The clone.py script only uses the data in the history database to populate the clone.
The history.py script receives the WAL replay log from the slave database as standard input.
It uses a couple of classes to consume and parse the log, \textit{Worker} and \textit{RegExer}.
The Worker class reads the log and keeps a list of all heap records it encounters.
When it reads a commit record it creates a new history version in the history database and runs through the heap records, saving the changes to the new history version.
To parse the log records it uses the RegExer class, which is a wrapper around regular expressions which are used to match and extract data from the replay log.
See chapter \ref{ch:reading-the-wal} for an explanation of the WAL replay log, heap records and commit records.

All SQL and PL/pgSQL code in Marty is part of the Python scripts.
Marty is not a large project and the current version contains a little over 1100 lines of code.
Included in that number is the file that contains the patch to the Postgres source code.

The source code for the current version of Marty can be found in appendix \ref{ch:appendix-source-code}.