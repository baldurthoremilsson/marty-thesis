\subsection{Schema information}
To initialize the history database the administrator runs the script \textit{history.py}.
It starts by creating the \textit{schema information tables} which Marty uses to store information about the schema of the master database:

\begin{description}
  \item[marty\_schemas]
    Contains the name and ID of all schemas in the slave database.
  \item[marty\_tables]
    Contains the name and ID of the tables in the slave database and a reference to the schema they are in.
    It also stores the \textit{internal name} which is used for the data table in the history and clone databases.
  \item[marty\_columns]
    Contains the name, number, type and length of each column in the tables.
    It also stores an internal name for the column that is used in the data tables in the history and clone databases.
\end{description}

Tables \ref{tbl:marty-schemas} to \ref{tbl:marty-columns} show information about the columns of these tables.
Figure \ref{fig:schema-information-example} shows an example of their contents.

\begin{table}[h]
  \centering
  \textbf{marty\_schemas}
  \begin{tabularx}{\textwidth}{llX}
    \textit{Column} & \textit{Type} & \textit{Description} \\
    \midrule
    \_ctid & tid & A reference to the pg\_namespace table in the slave database \\
    oid & oid & The ID of the schema in the slave database \\
    name & name & The name of the schema \\
    start & integer & First version where this schema is present in the database \\
    stop & integer & First version where this schema stops being present in the database \\
  \end{tabularx}
  \caption{The columns of the marty\_schemas table}
  \label{tbl:marty-schemas}
\end{table}

\begin{table}[h]
  \centering
  \textbf{marty\_tables}
  \begin{tabularx}{\textwidth}{llX}
    \textit{Column} & \textit{Type} & \textit{Description} \\
    \midrule
    \_ctid & tid & A reference to the pg\_class table in the slave database \\
    oid & oid & The ID of the table in the slave database \\
    name & name & The name of the table \\
    schema\_oid & oid & A reference to the schema that this table belongs to \\
    internal\_name & name & The name of the data table, see chapter \ref{ch:implementation-history-data} \\
    start & integer & First version where this table is present in the database \\
    stop & integer & First version where this table stops being present in the database \\
  \end{tabularx}
  \caption{The columns of the marty\_tables table}
  \label{tbl:marty-tables}
\end{table}

\begin{table}[h]
  \centering
  \textbf{marty\_columns}
  \begin{tabularx}{\textwidth}{llX}
    \textit{Column} & \textit{Type} & \textit{Description} \\
    \midrule
    \_ctid & tid & A reference to the pg\_attribute table in the slave database \\
    table\_oid & oid & A reference to the table this column is in \\
    name & name & The name of the column \\
    number & int2 & The index of this column in the table (is it the first, second, third etc.) \\
    type & name & The type of the column (int, text, boolean etc.) \\
    internal\_name & name & The name of this column in the data table, see chapter \ref{ch:implementation-history-data} \\
    start & integer & First version where this column is part of the table \\
    stop & integer & First version where this column stops being part of the table \\
  \end{tabularx}
  \caption{The columns of the marty\_tables table}
  \label{tbl:marty-columns}
\end{table}

Postgres uses \textit{system catalogs} to store information about the schema of its databases.
They are used internally, e.g. when Postgres reads from or writes to tables.
Information such as which file in the database cluster represents which table and the column names and types of each table are stored in the system catalogs.
They are ordinary tables which are stored in a schema called \textit{pg\_catalogs} and can by queried by a user just as any other table.

Marty reads information from four system catalogs in the slave database:

\begin{description}
  \item[pg\_namespace]
    Contains information about the schemas in the slave database.
    Information from this catalog is saved in \textit{marty\_schemas} in the history database. % TODO saved in or written to?
  \item[pg\_class]
    Contains information about the tables in the slave database.
    Information from this catalog is saved in \textit{marty\_tables} in the history database.
  \item[pg\_attribute]
    Contains information about the columns of the tables in the slave database.
    Information from this catalog is saved in \textit{marty\_columns} in the history database.
  \item[pg\_type]
    Contains additional information about the columns.
    Marty reads the name of the column type from this table (integer, text etc.) and stores it in \textit{marty\_columns} along with the other column related information.
\end{description}

The history.py script creates the schema information tables and then it populates them with information about the schema of the slave database.
This is of course also the schema of the master database so the history database really contains information about the master.

Marty inspects the schema of the slave database before any WAL records have been applied to it.
It writes the schema information to the history database, which creates the first history version.
Marty then starts the WAL replay in the slave database and reads the replay log.

When a new schema is created the WAL contains an \textit{insert} heap record for the pg\_namespace table.
Marty sees this in the replay log and queries the slave about this new schema.
It then creates a new history version that includes it.
The same happens when a new table is created or a new column is added to a table; a new history version is created that includes the new table or column.

If a schema, table or column is altered, e.g. when they are renamed, the WAL has an \textit{update} heap record.
Similarly when a schema, table or column is dropped the WAL has a \textit{delete} heap record.
Marty sees this in the replay log and creates new history versions with the altered database schema.

\begin{figure}[h!]
  \centering
    \includegraphics[width=0.75\textwidth]{schema-information-example}
  \caption{Example of the contents of the schema information tables}
  \label{fig:schema-information-example}
\end{figure}

See figure \ref{fig:schema-information-example} for an example of the contents of the schema information tables.
In the example the master database has one table, \textit{persons}, with two columns, \textit{age} of type integer and \textit{name} of type text.
The history part of the examples shows the contents of the schema information tables.

% TODO add reference to CTID and Versions subsections