\chapter{Introduction}
\pagenumbering{arabic}
\setcounter{page}{1}
Database management systems (DBMS) are used as datastores in many different systems in various fields.
They are rarely used as standalone products and are typically used to store data from other applications.
These applications are often in constant development with short development cycles, which include both manual and automated testing.
Those  tests are often run against datasets that are created to test for specific conditions and ideally they help with catching all bugs before they enter production. However, many projects can benefit from tests that are run against data from the production environment, either to complement the testing datasets or to provide data to test against in situations where no testing datasets exist.
The main disadvantage of using production data in testing is that cloning a large database can take a long time which slows down testing and development and it adds an overhead to the database in production which can have negative effects on the performance of the application in the production environment.

The goal of Marty is to offer a convenient and relatively efficient way to run tests for applications that use the PostgreSQL (Postgres) DBMS against live data on the production servers without adding overhead to them.
This is achieved by creating a testing database with empty tables that are populated  when they are first queried.
This saves time as only the tables which are used in the tests are populated and no time is spent copying the data for the other tables, which remain empty.
The data is not copied directly from the production server but from another instance of Postgres that stores a copy of the production data.
This ensures the consistency of the data in the cloned databases and it also minimizes the load on the production database.

The architecture of Marty enables users to inspect the state of the production database as it was at a certain point in time in the past.
This is similar to \textit{time travel} which was once a part of Postgres but was removed due to performance and storage space issues.
This can be beneficial in situations where the state of the database caused anomalies or bugs in the application, bugs which have since stopped because the state of the database has changed.
The user could then run the application with data from different points in time to debug it.

\section{Goals and purpose of Marty}
The goal of the development of Marty is to create an application that enables its users to clone a running database quickly.
The original idea was that software developers and testers would be able to clone a database that is used in production and stores large amounts of data that would normally take a considerable time to copy to another server.
Marty will speed up development and testing by reducing the time it takes to clone the production database and also uses techniques that reduce or prevent any negative impact that the cloning would have on the performance of the production database. 

Although the initial idea was for production databases to be cloned there is nothing that prevents Marty to be used for other kinds of databases, such as databases that are dedicated to storing test datasets that never enter production, as long as those databases fulfill the requirements for Marty.

The emphasis in the design and development of Marty is to minimize the time from when the cloning of a database is initialized and until the newly created clone can be used for testing.
The performance of the newly created database has not been a high priority as it is not intended to be used in a performance critical environment.
Thus Marty is not a solution that should be used to create clones of a database that are to be used for load balancing or failover or serve any other role in the production environment of an application.

Marty is supposed to be used in an environment where a one or a few databases need to be cloned regularly.
The architecture that was chosen for Marty requires a system administrator to set up and configure Marty for the environment where it is used.
This involves running a dedicated Postgres instance that is used as a reference when the clones are created and also configuring the production server to work with this dedicated instance in a certain way, which might require the production server to be restarted.
It should therefore be clear that Marty is not suited for cloning a database that only needs to be cloned for a limited number of times.
It should be most useful when the database to be cloned is large enough that the time saved by using Marty justifies the initial setup.
\section{Similar solutions}
The development of Marty was started in part to solve a problem that did not have any solutions available.
When a user wanted to replicate a Postgres database she needed to copy the whole database.
Tools like \textit{pg\_dump} exists to aid with database replication but any optimizations had to be created manually for each setup.
If the user wanted to keep some tables empty or skip them completely she needed to create her own solution, such as a script, that implemented that behaviour.
Postgres does not support lazy-loading of data natively like the clone databases in Marty require.

Heroku, a web hosting company, offers Postgres database hosting\footnote{https://www.heroku.com/postgres}.
It has implemented a feature called \textit{forking} which is based on the same idea as Marty but implemented differently.
The documentation for forking includes:

\textit{Preparing a fork can take anywhere from several minutes to several hours, depending on the size of your dataset\footnote{https://devcenter.heroku.com/articles/heroku-postgres-fork}.}

The fork is a complete replica that contains all the tables and database objects of the original database as well as all of its data.
This is unlike the way that Marty creates its clone databases; their tables start empty and are not populated with data until they are queried.
In that sense the forking of Postgres databases in Heroku serves a different purpose than Marty, which puts emphasis on the short initialization time of the clone databases.

There exist numerous clustering solutions for Postgres, such as \textit{Slony-I}\footnote{http://slony.info} and \textit{pgpool-II}\footnote{http://www.pgpool.net/mediawiki/index.php/Main\_Page}.
Many of these solutions can possibly be tailored to suit the needs of developers and testers who need replicas of the production database.
They are, however, not developed with this use case in mind so their usage for this situation can be problematic and can include much configuration and setup, if they can be used at all.
Marty is developed for a specific use case and is tailored to satisfy the requirements of that use case.
This makes it a better choice in the situation where developers and testers need to be able to quickly create replicas of the production database.

Software development includes testing in various stages of the development.
Various methods are used to test the software, such as automated unit tests and integration tests and manual testing that is performed by the developer or a dedicated tester.
It should be possible to use Marty for all stages of software testing.
The creation and initialization of the clone databases should be quick enough to make it feasible to create a new one for each feature that is being developed or tested.
This includes creating a new clone database for each unit test and integration test that is executed, even if there are hundreds of them and they are executed multiple times every day.
\section{Postgres}
Postgres\footnote{http://www.postgresql.org} is an SQL based DBMS that originated at the University of California, Berkely in the 1980's.
It was based on another DBMS, Ingres, and was released as a free and open source software in 1995.
It is developed by a global community under the name PostgreSQL Global Development Group, with a core team consisting of a half a dozen members and a large number of other contributors.
It is written in C and runs on multiple platforms.

Postgres is very mature and has a large number of features, including conformance with a large part of the SQL standard and a support for extension modules.
Many modules have been created to add new data types, offer new scripting languages for stored procedures and add functionality for specific types of data, such as geographical information.
It has a very extensive documentation and an active community that offers support for users through mailing lists and IRC channels.

Many companies offer commercial support and products based on Postgres with many more using it as a part of their internal systems.
It is used by government organizations and universities and many free and open source projects.

\subsection{Technical information}
There are a few implementation details that are used in this thesis for the discussion of the architecture and implementation of Marty.
Postgres also sometimes uses terminology to describe objects or ideas that is different from the one that is commonly is use.

A \textit{database cluster} in Postgres is a directory in the file system of the server that runs the Postgres instance\footnote{http://www.postgresql.org/docs/9.3/static/creating-cluster.html}.
This directory contains files and subfolders that store all contents of every database that runs on that instance of Postgres.
The files contain binary data that is generally not readable by programs other than the Postgres DBMS.

A \textit{relation} is a database object that stores some data.
One example of a relation is an ordinary table that is created in a Postgres database.
Its contents are stored in the database cluster directory in a \textit{relation file}.
This file consists of \textit{blocks} of data, each of which contains one or many \textit{tuples} that store the values of the relation.
Other types of relations are e.g. views, sequences and foreign tables.

Postgres keeps a log of all changes that are made to the relation files, and other files and directories in the database cluster, in a so-called \textit{write-ahead log} or WAL\footnote{http://www.postgresql.org/docs/9.3/static/wal-intro.html}.
The WAL contains the binary records that Postgres inserts into the files in the cluster.
It is used for recovery after a server crash and can also be used to replicate the database in another instance of Postgres.
Marty uses the WAL to inspect the changes that have been made to the master database which the clone databases replicate.

Postgres is a \textit{transactional} database.
This means that every operation that a user executes in the database is wrapped in a transaction.
Inside a transaction the state of the database is unaffected by the changes that are made in other transactions, even if they run concurrently.
This is important to make it possible for users to query the database even though another user is updating the same tables at the same time.
Postgres implements this by using \textit{multiversion concurrency control} (MVCC)\footnote{http://www.postgresql.org/docs/9.3/static/mvcc.html}.
It enables a database to contain multiple versions of the same data at the same time to ensure that the data in the tables in each transaction is correct and consistent.

The details of how Postgres implements the MVCC are complex, but the most relevant part for the description of Marty are the \textit{xmin} and \textit{xmax} columns.
These are hidden system columns which Postgres adds to every table.
They are used to define which transactions can see which rows in the table.
Every transaction that runs on the server has a \textit{transaction ID}.
The ID of the transaction that inserts a row into a table is recorded in the xmin column.
When a row is deleted the ID of the transaction that deleted it is recorded in the xmax column, until then it contains NULL.
A transaction that updates the values of a row actually inserts a new row with the updated values and marks the old row as deleted.
A row is part of a transaction if the equation \textit{xmin ≤ transaction ID < xmax} holds.

Marty uses a similar technique when it stores multiple versions of the data from the master database.
\section{Thesis Overview}
The rest of the tesis is organized as follows: Chapter \ref{ch:architecture} contains a description of the architecture of Marty and explains the purpose of each part of the system.
Chapter \ref{ch:implementation} contains a detailed description of the implementation of each part of the system, with references and explanations to the relevant parts of Postgres.
Chapter \ref{ch:current-status} contains the conclusion of the thesis along with a description of the limitations of the current design and ideas for future work on Marty.
The complete source code for the current version of Marty is incuded in Appendix \ref{ch:appendix-source-code}.
Appendix \ref{ch:appendix-name} explains the origins of the name for the project.

