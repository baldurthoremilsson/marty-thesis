\section{Postgres}
Postgres\footnote{http://www.postgresql.org} is an SQL based DBMS that originated at the University of California, Berkely in the 1980's.
It was based on another DBMS, Ingres, and was released as a free and open source software in 1995.
It is developed by a global community under the name PostgreSQL Global Development Group, with a core team consisting of a half a dozen members and a large number of other contributors.
It is written in C and runs on multiple platforms.

Postgres is very mature and has a large number of features, including conformance with a large part of the SQL standard and a support for extension modules.
Many modules have been created to add new data types, offer new scripting languages for stored procedures and add functionality for specific types of data, such as geographical information.
It has a very extensive documentation and an active community that offers support for users through mailing lists and IRC channels.

Many companies offer commercial support and products based on Postgres with many more using it as a part of their internal systems.
It is used by government organizations and universities and many free and open source projects.

\subsection{Technical information}
There are a few implementation details that are used in this thesis for the discussion of the architecture and implementation of Marty.
Postgres also sometimes uses terminology to describe objects or ideas that is different from the one that is commonly is use.

A \textit{database cluster} in Postgres is a directory in the file system of the server that runs the Postgres instance\footnote{http://www.postgresql.org/docs/9.3/static/creating-cluster.html}.
This directory contains files and subfolders that store all contents of every database that runs on that instance of Postgres.
The files contain binary data that is generally not readable by programs other than the Postgres DBMS.

A \textit{relation} is a database object that stores some data.
One example of a relation is an ordinary table that is created in a Postgres database.
Its contents are stored in the database cluster directory in a \textit{relation file}.
This file consists of \textit{blocks} of data, each of which contains one or many \textit{tuples} that store the values of the relation.
Other types of relations are e.g. views, sequences and foreign tables.

Postgres keeps a log of all changes that are made to the relation files, and other files and directories in the database cluster, in a so-called \textit{write-ahead log} or WAL\footnote{http://www.postgresql.org/docs/9.3/static/wal-intro.html}.
The WAL contains the binary records that Postgres inserts into the files in the cluster.
It is used for recovery after a server crash and can also be used to replicate the database in another instance of Postgres.
Marty uses the WAL to inspect the changes that have been made to the master database which the clone databases replicate.

Postgres is a \textit{transactional} database.
This means that every operation that a user executes in the database is wrapped in a transaction.
Inside a transaction the state of the database is unaffected by the changes that are made in other transactions, even if they run concurrently.
This is important to make it possible for users to query the database even though another user is updating the same tables at the same time.
Postgres implements this by using \textit{multiversion concurrency control} (MVCC)\footnote{http://www.postgresql.org/docs/9.3/static/mvcc.html}.
It enables a database to contain multiple versions of the same data at the same time to ensure that the data in the tables in each transaction is correct and consistent.

The details of how Postgres implements the MVCC are complex, but the most relevant part for the description of Marty are the \textit{xmin} and \textit{xmax} columns.
These are hidden system columns which Postgres adds to every table.
They are used to define which transactions can see which rows in the table.
Every transaction that runs on the server has a \textit{transaction ID}.
The ID of the transaction that inserts a row into a table is recorded in the xmin column.
When a row is deleted the ID of the transaction that deleted it is recorded in the xmax column, until then it contains NULL.
A transaction that updates the values of a row actually inserts a new row with the updated values and marks the old row as deleted.
A row is part of a transaction if the equation \textit{xmin ≤ transaction ID < xmax} holds.

Marty uses a similar technique when it stores multiple versions of the data from the master database.