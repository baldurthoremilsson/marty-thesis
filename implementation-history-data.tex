\subsection{Data tables}
The history database stores a copy of all the data in the tables from the slave database.
Marty creates a table in the history database for each table in the slave.
This happens right after Marty has saved the information about the original table to the history database.
These tables are the \textit{data tables} of the history database.

The schema of a data table is not identical to the schema of the original table in the slave database.
Marty adds three columns for metadata; \textit{\_ctid}, \textit{start} and \textit{stop}.
These columns are also present in the schema information tables where they serve the same purpose, see chapters \ref{ch:implementation-history-ctid} and \ref{ch:implementation-history-versions} for more information.

When a column is added to the original table in the slave database Marty also adds it to the data table.
If a column is dropped from the original table Marty still keeps it in the data table.
This is necessary because Marty keeps old versions of the data in the table and must therefor keep all columns that have been added to the original table.

When a row is deleted from the original table Marty marks it as deleted in the data table.
The row is still kept in the data table as part of an old history version because the data must still be accessible.
When a row is updated in the original table Marty inserts a new row into the data table with the updated values and marks the old row as deleted.
This behaviour is similar to the \textit{multiversion concurrency control} (MVCC) that Postgres uses to allow more than one transaction to use the same table at the same time.

Marty does not copy any constraints from the original tables to the data tables.
The only data that the data tables contain has been copied from the original tables in the slave database.
This data confirms to the constraints of the original tables and thus it is unnecessary to replicate the constraints in the data tables.
The constraints would also cause trouble, e.g. when Marty stores updated rows from a table with a unique constraint.
If the values of the unique columns are \textit{not} updated but only values in other columns then the data table could not store both the old and the new versions of the rows.
For these reasons Marty does not copy any constraints from the original tables.

