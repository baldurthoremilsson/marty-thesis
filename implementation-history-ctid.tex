\subsection{CTID columns}
\label{ch:implementation-history-ctid}
The schema information tables have a column called \textit{\_ctid} and the data tables have an identical column called \textit{data\_ctid}.
The values in these columns reference the values in the \textit{ctid} columns in the corresponding tables in the slave database.
For the schema information table these are the system catalogs and for the data tables the corresponding tables are the original tables in the slave database.

The ctid column is a hidden system column that Postgres adds to all tables when they are created.
It is of type \textit{tid}, or tuple identifier, which is a pair of numbers that identify the physical location of the row inside the relation file in the database cluster.
The relation files are made up of \textit{blocks} which store the contents of the relations.
Each block contains one or more \textit{tuples} which store the actual data of the table rows.
The tuple identifier consists of a \textit{block index} and a \textit{tuple index}.
The block index is zero-based while the tuple index is one-based so the first tuple ID is (0,1).

\begin{lstlisting}[caption={WAL update and delete example},label={lst:wal-update-delete},numbers=left,xleftmargin=2em]
LOG:  REDO @ 0/60080CC; LSN 0/6008198: prev 0/600806C; xid 735; len 37; bkpb0: Heap - update: rel 1663/16384/16385; tid 0/1 xmax 735 ; new tid 0/2 xmax 0
LOG:  REDO @ 0/600822C; LSN 0/6008264: prev 0/6008204; xid 737; len 26: Heap - delete: rel 1663/16384/16385; tid 0/3 KEYS_UPDATED
\end{lstlisting}

The values in the ctid column can be used to identify each row in the database.
Marty saves this value in the schema information tables and data tables to be able to identify deleted and updated rows.
The WAL replay log from the slave instance contains the tuple identifier of the rows that are updated or deleted in the WAL records.
Marty uses the tid value to locate the correct row in the schema information tables or data tables in the history database and marks these rows as deleted.
See Listing \ref{lst:wal-update-delete} for an example.
In the example the update record in the first line inserts a new row with tid value of (0,2) to replace the old values in row with tid value of (0,1).
The delete record in the second line deletes a row with tid value of (0,3).

Note that the value of the ctid column can change, e.g. when Postgres \textit{vacuums} the table or when a row is updated.
Therefore it is not a viable long-term key to identify the row.
This does not affect the way Marty uses it because it only needs to know the tid value of the row until it changes.
When it is changed in the slave database Marty reflects this change in the history database and saves the new tid value.

Figures \ref{fig:schema-information-example} and \ref{fig:history-data-table-contents} show an example of the values in the \textit{\_ctid} and \textit{data\_ctid} columns in the history database.