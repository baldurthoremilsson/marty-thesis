\subsection{History versions}
\label{ch:implementation-history-versions}
The history database stores multiple versions of the data from the slave database.
This is necessary because of the delayed population of the tables in the clone databases.
From the time when a clone is initialized and until its tables are populated with data the slave database can change, causing inconsistency in the clone.
Therefor the clone must have access to the data as it was when the user initialized the clone database.

When the history database is initialized Marty creates the table \textit{marty\_updates}, see table \ref{tbl:marty-updates} for a list of its columns.
It stores the ID of each version along with two timestamps, the local time and the master time.
The local timestamp (in the column \textit{time}) is the time when this version was created in the history database.
The master time is the time when the transaction that created this version was executed on the master database.
The ID is a simple counter that starts in 1 and increments by one for each new version.

\begin{table}[h]
  \centering
  \textbf{marty\_updates}
  \begin{tabularx}{\textwidth}{llX}
    \textit{Column} & \textit{Type} & \textit{Description} \\
    \midrule
    id & serial & Primary key - a unique ID for each history version \\
    time & timestamp & The date and time when this version was created in the history database \\
    mastertime & timestamp & The date and time of the transaction that created this version on the master database \\
  \end{tabularx}
  \caption{ The columns of the marty\_updates table}
  \label{tbl:marty-updates}
\end{table}

The schema information tables and data tables in the history database have two columns that tell Marty which versions a particular row is part of.
These columns are \textit{start} and \textit{stop} and they are foreign keys that reference the version ID the the marty\_updates table.
See tables \ref{tbl:marty-schemas} to \ref{tbl:marty-columns} for a list of columns in the schema information tables and figure \ref{fig:history-data-table-2} for an example of the columns in a data table.

When a row in inserted into one of the tables in the history database it is part of the current history version.
Marty inserts the ID of the current version into the \textit{start} column to reflect this.
When a row is deleted or updated in the slave database the corresponding row in the history database is marked as deleted.
Marty does this by updating the value in the \textit{stop} column of the deleted row to the current history version ID.
An example of this is a row where the start ID is 3 and the stop ID is 6.
This row is part of versions 3, 4 and 5.
A row that has not been deleted or updated has \textit{null} in the stop column.
If the current version ID is 10 and a row has start ID 7 and stop ID null then this row is part of versions 7, 8, 9 and 10 and all future versions until it is marked as deleted.

This method is similar to the way Postgres implements \textit{multiversion concurrency control} (MVCC).
The MVCC is used to allow more than one transaction to use the same table at the same time.
Postgres creates hidden system columns on all tables called \textit{xmin} and \textit{xmax}.
They store the \textit{transaction ID} of the transactions that created the row and deleted or updated it, respectively.
This is similar to the version IDs and the way Marty uses the start and stop columns.
Marty does not, however, use the values in the xmin and xmax columns to determine the version numbers for the rows in the history database.
Instead the WAL replay log is used to inspect when each row is created and when it is deleted.

When the history database is initialized Marty inspects the slave database and creates the first history version.
It contains all schemas and tables in the slave database before any WAL records have been applied to it.
When Marty is finished inspecting the slave it starts the WAL replay.
The slave pauses the replay when the first transaction has been applied and Marty creates the second version with all the changes from that transaction.
Marty then restarts the WAL replay and repeats the process for each new transaction.